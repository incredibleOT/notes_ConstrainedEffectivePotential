Since it turned out, that the approach where one assumes to be in te broken phase with the potential given in~\eqref{eq:CEP_lowerBound_with_phi_6} works pretty well in 
determining the phase structure in presence of small but negative  $\lambda$ and small positive $\lambda_6$ we continue from this approach. Since there are still 
deviation between simulation data and prediction from the CEP, David suggested, to use a similar approach as in chapter~\ref{ch:alternativeExpansion}, where the gaussian 
contributions from the interaction part are actually considered for the gaussian part with the drawback of having a vev-dependent bosonic determinant and more complicated 
expressions for the propagators. However, in this ansatz no iterative scheme is needed, since the Higgs mass should only be taken from the curvature of the potential in its
minimum. Still, the minimization has to be performed, which will be slightly more complicated.

Starting point here is the bosonic action, where the scalar field was already decomposed into Higgs and Goldstone modes. The only difference in treating those two 
kinds of fields is in the assumtion of the zero mode. For the Higgs field it is proportional to the vev, while it is zero for the Goldstones:
\begin{equation}\label{eq:def_zeromodes_inBrokenPhase}
 \tilde h_0 = \sqrt{V} \hat v,\quad\quad \tilde g_0^{\alpha}=0.
\end{equation}
The action is:
\begin{align}\label{eq:bosonic_action_in_broken_phase_momentum}
 S & = \frac{1}{2}\sum\limits_{p} \left \{ \tilde h_{-p} \left( \hat p^2 + m_0^2 \right) \tilde h_{p} + 
                \sum\limits_{\alpha} \tilde g^{\alpha}_{-p} \left( \hat p^2 + m_0^2 \right) \tilde g^{\alpha}_{p} \right\}
        \nonumber \\
    & + \frac{\lambda}{V} \sum \limits_{p_1 \dots p_4} \delta_{p_1+ \dots +p_4,0} 
                \left( \tilde h_{p_1} \tilde h_{p_2} + \sum\limits_{\alpha} \tilde g^{\alpha}_{p_1} \tilde g^{\alpha}_{p_2} \right)
                \left( \tilde h_{p_3} \tilde h_{p_4} + \sum\limits_{\alpha} \tilde g^{\alpha}_{p_3} \tilde g^{\alpha}_{p_4} \right)
             \nonumber \\
    & + \frac{\lambda_6}{V^2} \sum \limits_{p_1 \dots p_6} \delta_{p_1+ \dots +p_6,0} 
                \left( \tilde h_{p_1} \tilde h_{p_2} + \sum\limits_{\alpha} \tilde g^{\alpha}_{p_1} \tilde g^{\alpha}_{p_2} \right)
                \left( \tilde h_{p_3} \tilde h_{p_4} + \sum\limits_{\alpha} \tilde g^{\alpha}_{p_3} \tilde g^{\alpha}_{p_4} \right)
                \left( \tilde h_{p_5} \tilde h_{p_6} + \sum\limits_{\alpha} \tilde g^{\alpha}_{p_5} \tilde g^{\alpha}_{p_6} \right).
\end{align}

With this, a decomposition of this action can be done by decomposing it into a tree-level, a gaussian and an interaction part. Tree-level is easy, simply collect
all terms that only contain zero modes. 
\begin{equation}\label{eq:tree_level_action_inBrokenPhase}
 S_B^{\text{tree}} = V \left( \frac{m_0^2}{2} \hat v^2 + \lambda \hat v^4 + \lambda_6 \hat v ^6 \right).
\end{equation}
For th gaussian term, we collect all terms, where all but two fields are set to their zero modes. This should lead to:
\begin{equation}\label{eq:improvedGaussian_action_inBrokenPhase}
 S_B^{\text{gauss}} = \frac{1}{2}\sum\limits_{p \neq 0} \left\{
                 \tilde h_{-p} \left ( \hat p^2 + m_0^2 + 12 \hat v^2 \lambda + 30 \hat v^4 \lambda_6 \right) \tilde h_p
                 + \sum\limits_{\alpha} 
                 \tilde g^{\alpha}_{-p} \left ( \hat p^2 + m_0^2 + 4 \hat v^2 \lambda + 6 \hat v^4 \lambda_6 \right) \tilde g^{\alpha}_p
                 \right\}
\end{equation}
The interaction part is simply given by the rest:
\begin{align}\label{eq:improvedInteraction_action_inBrokenPhase}
 S_B^{\text{int}} & = \frac{\lambda}{V}\widetilde{\sum\limits_{p_1\dots p_4}} \delta_{p_1+ \dots +p_4,0} 
                      \left( \tilde h_{p_1} \tilde h_{p_2} + \sum\limits_{\alpha} \tilde g^{\alpha}_{p_1} \tilde g^{\alpha}_{p_2} \right)
                      \left( \tilde h_{p_3} \tilde h_{p_4} + \sum\limits_{\alpha} \tilde g^{\alpha}_{p_3} \tilde g^{\alpha}_{p_4} \right)
                      \nonumber \\
                  & + \frac{\lambda_6}{V}\hat v^2 \widetilde{\sum\limits_{p_1\dots p_4}} \delta_{p_1+ \dots +p_4,0} \left\{
                    + 30 \, \tilde h_{p_1} \tilde h_{p_2} \tilde h_{p_3} \tilde h_{p_4} 
                    + 18 \, \tilde h_{p_1} \tilde h_{p_2} \sum\limits_{\alpha} \tilde g^{\alpha}_{p_3} \tilde g^{\alpha}_{p_4}
                    + 3  \, \sum\limits_{\alpha,\beta} \tilde g^{\alpha}_{p_1} \tilde g^{\alpha}_{p_2}\tilde g^{\beta}_{p_3} \tilde g^{\beta}_{p_4}
                 \right\}
                      \nonumber \\
                  & + \frac{\lambda_6}{V^2} \widetilde {\sum \limits_{p_1 \dots p_6}} \delta_{p_1+ \dots +p_6,0} 
                \left( \tilde h_{p_1} \tilde h_{p_2} + \sum\limits_{\alpha} \tilde g^{\alpha}_{p_1} \tilde g^{\alpha}_{p_2} \right)
                \left( \tilde h_{p_3} \tilde h_{p_4} + \sum\limits_{\alpha} \tilde g^{\alpha}_{p_3} \tilde g^{\alpha}_{p_4} \right)
                \left( \tilde h_{p_5} \tilde h_{p_6} + \sum\limits_{\alpha} \tilde g^{\alpha}_{p_5} \tilde g^{\alpha}_{p_6} \right).
\end{align}
From the gaussian part~\eqref{eq:improvedGaussian_action_inBrokenPhase}, we get the propagators:
\begin{equation}
    \label{eq:prop_HiggsAndGoldstone_improved_inBrokenPhase}
 \contraction{}{\tilde h}{{}_p}{\tilde h{}}  \tilde h_p \tilde h_q = 
                              \frac{ \delta_{p+q,0}}{\hat p^2 + m_0^2 + 12 \hat v^2 \lambda + 30 \hat v^4 \lambda_6 } \quad \text{and} \quad
 \contraction{}{\tilde g}{{}^{\alpha}_p}{\tilde g{}}  \tilde g^{\alpha}_p \tilde g^{\beta}_q = 
                              \frac{ \delta_{p+q,0} \delta_{\alpha,\beta}}{\hat p^2 + m_0^2 + 4 \hat v^2 \lambda + 6 \hat v^4 \lambda_6 }.
\end{equation}
As before, for convenience let's define the propagator sums for later use:
\begin{equation}\label{eq:propagatorSum_improved_inBrokenPhase}
\tilde P_H = \frac{1}{V} \sum\limits_{p\neq 0} \frac{1}{ \hat p^2 + m_0^2 + 12 \hat v^2 \lambda + 30 \hat v^4 \lambda_6 } \quad\text{and}\quad
\tilde P_H = \frac{1}{V} \sum\limits_{p\neq 0} \frac{1}{ \hat p^2 + m_0^2 + 4 \hat v^2 \lambda + 6 \hat v^4 \lambda_6 }.
\end{equation}

In this approach, the bosonic determinant has to be considered for the CEP as well as the first order contributions in $\lambda$ and $\lambda_6$
\begin{align}
      \label{eq:CEP_improved_inBrokenPhase_all}
 U(\hat v) & = U^F(\hat v) + U^T(\hat v) + U^D(\hat v) + U^{(1)}(\hat v) 
            \\
      \label{eq:CEP_improved_inBrokenPhase_Fermion}
 U^F(\hat v) & = -\frac{2\, N_f}{V} \sum\limits_p \left\{
                  \log\left| \nu(p) + y_t \cdot \hat v \cdot \left( 1-\frac{1}{2 \rho} \right) \nu(p) \right|^2 +
                  \log\left| \nu(p) + y_b \cdot \hat v \cdot \left( 1-\frac{1}{2 \rho} \right) \nu(p) \right|^2 \right\}
            \\
      \label{eq:CEP_improved_inBrokenPhase_Tree}
 U^T(\hat v) & = \frac{m_0^2}{2} \hat v^2 + \lambda \hat v^4 + \lambda_6 \hat v^6
             \\
      \label{eq:CEP_improved_inBrokenPhase_BosDet}
 U^D(\hat v) & = \sum\limits_{p \neq 0} \left\{
                  \log \left( \hat p^2 + m_0^2 + 12 \hat v^2 \lambda + 30 \hat v^4 \lambda_6 \right) + 
                  3\, \log \left( \hat p^2 + m_0^2 + 4 \hat v^2 \lambda + 6 \hat v^4 \lambda_6 \right) \right\}
\end{align}
