Here the addition of a higher dimensional operator should be discussed. The new bosonic action in the lattice notation is then:
\begin{equation}\label{eq:bosonicLatticeActionWithLambda_6}
 S_B[\Phi]  = -\kappa \sum\limits_{x,\mu} \Phi_x^{\dagger} \left[\Phi_{x+\hat\mu} + \Phi_{x-\hat\mu} \right] 
              + \sum\limits_x \Phi_x^{\dagger} \Phi_x 
              + \hat{\lambda} \sum\limits_x \left( \Phi_x^{\dagger} \Phi_x - N_f\right)^2 + \hat \lambda_6 \sum\limits_x \left( \Phi_x^{\dagger} \Phi_x \right)^3
\end{equation}
With that, the CEP changes. With the given notation however, only the tree level part changes, since the new parameter does not enter the gaussion contribution.
The Fourier transorm of the new term looks like:
\begin{align}
 \sum\limits_{x} \left( \Phi_x^{\dagger} \Phi_x \right)^3 &= 
                     \frac{1}{V^3}  \sum\limits_{x}  \sum\limits_{p_1,\dots p6}  
                     \left( e^{-i\, p_1 \cdot x}  \tilde\Phi_{p_1}^{\dagger} \right) \left( e^{i\, p_2 \cdot x}  \tilde\Phi_{p_2} \right)
                     \left( e^{-i\, p_3 \cdot x}  \tilde\Phi_{p_3}^{\dagger} \right) \left( e^{i\, p_4 \cdot x}  \tilde\Phi_{p_4} \right)
                     \left( e^{-i\, p_5 \cdot x}  \tilde\Phi_{p_5}^{\dagger} \right) \left( e^{i\, p_6 \cdot x}  \tilde\Phi_{p_6} \right), \nonumber \\
                  &= \frac{1}{V^3}  \sum\limits_{p_1,\dots p6} 
                     \sum\limits_{x}  e^{ -i (p_1 + p_3 + p_5 - p_2 - p_4 - p_6) \cdot x }  
                     \tilde\Phi_{p_1}^{\dagger} \tilde\Phi_{p_2} \tilde\Phi_{p_3}^{\dagger} \tilde\Phi_{p_4} \tilde\Phi_{p_5}^{\dagger} \tilde\Phi_{p_6},
                     \nonumber \\
                  &= \frac{1}{V^2}\sum\limits_{p_1,\dots p6} \delta_{p_1 + p_3 + p_5, p_2 + p_4 + p_6} 
                     \tilde\Phi_{p_1}^{\dagger} \tilde\Phi_{p_2} \tilde\Phi_{p_3}^{\dagger} \tilde\Phi_{p_4} \tilde\Phi_{p_5}^{\dagger} \tilde\Phi_{p_6}
                      \label{eq:FourierTrafoPhiToTheSix}.
\end{align}
The contribution do the tree-level of the CEP is then:
\begin{multline}\label{eq:CEP_tree_level_Contribution_from_lambda_6}
 U^{\text{tree}}(m_{\Phi},s_{\Phi}) = -8 \kappa \left( m_{\Phi}^2 - s_{\Phi}^2 \right)   +   \left( m_{\Phi}^2 + s_{\Phi}^2 \right) 
                         + \hat\lambda \left( m_{\Phi}^4 + s_{\Phi}^4 + 6 m_{\Phi}^2 s_{\Phi}^2 - 2 N_f \left(m_{\Phi}^2 + s_{\Phi}^2 \right) \right) \\
                         + \hat\lambda_6 \left( m_{\Phi}^6 + s_{\Phi}^6 + 15 m_{\Phi}^4 s_{\Phi}^2 + 15 m_{\Phi}^2 s_{\Phi}^4\right).
\end{multline}
\textbf{let this check!}
If one changes to the notation in Philipp's thesis, with the common factor $N_f$ drawn out, one has to reparametrize $\hat\lambda_6$:
\begin{equation}\label{eq:lambda_6_reparametrized}
 \hat \lambda_6 = \frac{\lambda_{6N}}{N_f^2},
\end{equation}
leading to an equivalent contribution to $\tilde U(\breve{m}_{\Phi},\breve{s}_{\Phi})$.

\subsection*{Higher diemensional operators and 1st order in $\hat\lambda_6$}

If one wants to compute the first order contribution in $\hat\lambda_6$, analogous to~\eqref{eq:CEP_oneLoopAddition}, one must consider all possible contractions.
As for the first order in $\hat \lambda$, the full contraction of all six fields does not contribute, since it is independent of $m_{\Phi}$ and $s_{\Phi}$. Since texing the 
contractions is annoying, the combinatorics can be found in tab.~\ref{tab:combFactors_1stOrder_lambda6}. \textbf{In addition to thoase combinatoric
factors a factor of 4 has to be taken into account for each contraction of momenta coming from the nominator in eq.~\eqref{eq:ContractionOfBosonicField}.}
With this, the contribution to the potential is:
\begin{align}\label{eq:CEP_1stOrderContribution_lambda6}
 U(m_{\Phi}, s_{\Phi}) & \rightarrow 
                          U(m_{\Phi}, s_{\Phi})  +  \hat \lambda_6\, \left[ 36\, \left(m_{\Phi}^4 + s_{\Phi}^4 + 6 m_{\Phi}^2 s_{\Phi}^2\right) P_B 
                          + 576 \left(m_{\Phi}^2 + s_{\Phi}^2 \right) P_B^2 \right],
        \\ \label{eq:CEP_1stOrderContribution_lambda6_rescaled}
 \tilde U(\breve{m}_{\Phi},\breve{s}_{\Phi}) & \rightarrow  
       \tilde U(\breve{m}_{\Phi},\breve{s}_{\Phi}) 
                          + \hat \lambda_6\, \left[ \frac{36}{N_f} \left(\breve m_{\Phi}^4 + \breve s_{\Phi}^4 + 6 \breve m_{\Phi}^2 \breve s_{\Phi}^2\right) P_B 
                          + \frac{576}{N_f^2} \left(\breve m_{\Phi}^2 + \breve s_{\Phi}^2 \right) P_B^2 \right],
\end{align}

\begin{table}[htb]\centering{
%   \rowcolors{1}{white}{lightgray}
  \begin{tabular}{| c | c | c | c || c | c | c | c |}
		\hline
		coefficient              & $p_1$ & $p_3$ & $p_5$ & $p_2$ & $p_4$ & $p_6$ & factor               \\ \hline
		$m_{\Phi}^2$             &  0    &  $p$  &  $q$  &   0   &  $p$  &  $q$  & $6\times6=36$        \\ \hline
		$s_{\Phi}^2$             & $p_s$ &  $p$  &  $q$  & $p_s$ &  $p$  &  $q$  & $6\times6=36$        \\ \hline
		$m_{\Phi}^4$             &   0   &  0    &  $p$  &   0   &   0   &  $p$  & $3\times3=9$         \\ \hline
		$s_{\Phi}^4$             & $p_s$ & $p_s$ &  $p$  & $p_s$ & $p_s$ &  $p$  & $3\times3=9$         \\ \hline
		$m_{\Phi}^2\,m_{\Phi}^2$ &   0   &  0    &  $p$  & $p_s$ & $p_s$ &  $p$  & $3\times3\times2=18$ \\ \hline
		$m_{\Phi}^2\,m_{\Phi}^2$ &   0   & $p_s$ &  $p$  &   0   & $p_s$ &  $p$  & $6\times6       =36$ \\ \hline
  \end{tabular}}
  \caption{Combinatoric factors for the inclusion of the first order contribution of $\hat\lambda_6$}\label{tab:combFactors_1stOrder_lambda6}
\end{table}


