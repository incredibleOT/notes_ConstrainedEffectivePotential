Here the addition of a higher dimensional operator should be discussed. The new bosonic action in the lattice notation is then:
\begin{equation}\label{eq:bosonicLatticeActionWithLambda_6}
 S_B[\Phi]  = -\kappa \sum\limits_{x,\mu} \Phi_x^{\dagger} \left[\Phi_{x+\hat\mu} + \Phi_{x-\hat\mu} \right] 
              + \sum\limits_x \Phi_x^{\dagger} \Phi_x 
              + \hat{\lambda} \sum\limits_x \left( \Phi_x^{\dagger} \Phi_x - N_f\right)^2 + \hat \lambda_6 \sum\limits_x \left( \Phi_x^{\dagger} \Phi_x \right)^3
\end{equation}
With that, the CEP changes. With the given notation however, only the tree level part changes, since the new parameter does not enter the gaussion contribution.
The Fourier transorm of the new term looks like:
\begin{align}
 \sum\limits_{x} \left( \Phi_x^{\dagger} \Phi_x \right)^3 &= 
                     \frac{1}{V^3}  \sum\limits_{x}  \sum\limits_{p_1,\dots p6}  
                     \left( e^{-i\, p_1 \cdot x}  \tilde\Phi_{p_1}^{\dagger} \right) \left( e^{i\, p_2 \cdot x}  \tilde\Phi_{p_2} \right)
                     \left( e^{-i\, p_3 \cdot x}  \tilde\Phi_{p_3}^{\dagger} \right) \left( e^{i\, p_4 \cdot x}  \tilde\Phi_{p_4} \right)
                     \left( e^{-i\, p_5 \cdot x}  \tilde\Phi_{p_5}^{\dagger} \right) \left( e^{i\, p_6 \cdot x}  \tilde\Phi_{p_6} \right), \nonumber \\
                  &= \frac{1}{V^3}  \sum\limits_{p_1,\dots p6} 
                     \sum\limits_{x}  e^{ -i (p_1 + p_3 + p_5 - p_2 - p_4 - p_6) \cdot x }  
                     \tilde\Phi_{p_1}^{\dagger} \tilde\Phi_{p_2} \tilde\Phi_{p_3}^{\dagger} \tilde\Phi_{p_4} \tilde\Phi_{p_5}^{\dagger} \tilde\Phi_{p_6},
                     \nonumber \\
                  &= \frac{1}{V^2}\sum\limits_{p_1,\dots p6} \delta_{p_1 + p_3 + p_5, p_2 + p_4 + p_6} 
                     \tilde\Phi_{p_1}^{\dagger} \tilde\Phi_{p_2} \tilde\Phi_{p_3}^{\dagger} \tilde\Phi_{p_4} \tilde\Phi_{p_5}^{\dagger} \tilde\Phi_{p_6}
                      \label{eq:FourierTrafoPhiToTheSix}.
\end{align}
The contribution do the tree-level of the CEP is then:
\begin{multline}\label{eq:CEP_tree_level_Contribution_from_lambda_6}
 U^{\text{tree}}(m_{\Phi},s_{\Phi}) = -8 \kappa \left( m_{\Phi}^2 - s_{\Phi}^2 \right)   +   \left( m_{\Phi}^2 + s_{\Phi}^2 \right) 
                         + \hat\lambda \left( m_{\Phi}^4 + s_{\Phi}^4 + 6 m_{\Phi}^2 s_{\Phi}^2 - 2 N_f \left(m_{\Phi}^2 + s_{\Phi}^2 \right) \right) \\
                         + \hat\lambda_6 \left( m_{\Phi}^6 + s_{\Phi}^6 + 15 m_{\Phi}^4 s_{\Phi}^2 + 15 m_{\Phi}^2 s_{\Phi}^4\right).
\end{multline}
\textbf{let this check!}
If one changes to the notation in Philipp's thesis, with the common factor $N_f$ drawn out, one has to reparametrize $\hat\lambda_6$:
\begin{equation}\label{eq:lambda_6_reparametrized}
 \hat \lambda_6 = \frac{\lambda_{6N}}{N_f^2},
\end{equation}
leading to an equivalent contribution to $\tilde U(\breve{m}_{\Phi},\breve{s}_{\Phi})$.


