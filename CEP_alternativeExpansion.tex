\label{ch:alternativeExpansion}
Here another way to expand the action should be discussed. More accurate: More terms will contribute to the Gaussian part of the action $S_{B,0}$ 
which might improve the predictive power of the CEP without the ened to include the one loop term. What will be done is: Take from the interactive part those
term, that only have two of the bosonic fields being not the zero or staggered mode. With that, they can be considered as contribution to the 
Gaussioan part. The drawback from this is, that one has to include the bosonic determinant, since it then depends on $m_{\Phi}$ and $s_{\Phi}$. Further,
if one considers further orders in $\hat\lambda$, the propagator sums will be more complicated and also vacuum bubbles are not independent of $m_{\Phi}$ and $s_{\Phi}$
anymore.

Starting point is the expression for the $\hat\lambda \left( \Phi^{\dagger}\Phi \right)^2$ term. So far, the decomposition of this term looked like:
\begin{equation}\label{eq:naive_decomposition_of_phi4_term}
 \frac{\hat\lambda}{V} \sum\limits_{p,q,r,s} \delta_{p+r,q+s} \tilde\Phi_p^{\dagger} \tilde\Phi_q \tilde\Phi_r^{\dagger} \tilde\Phi_s 
    = V\left( \hat\lambda\left( m_{\Phi}^4 + s_{\Phi}^4 + 6 m_{\Phi}^2s_{\Phi}^2 \right) \right)
    + \frac{\hat\lambda}{V} \widehat{\sum\limits_{p,q,r,s}}\delta_{p+r,q+s} \tilde\Phi_p^{\dagger} \tilde\Phi_q \tilde\Phi_r^{\dagger} \tilde\Phi_s,
\end{equation}
with the hat above the sum indicating that not \textit{all} occuring momenta are eigther $0$~and/or~$p_s$. The terms in the sum, where two of the momenta 
are eigther $0$ or $p_s$ are then taken care of in the first order expansion of $\hat\lambda$.
Another possibility would be:
\begin{align}\label{eq:alternative_decomposition_of_phi4_term}
 \frac{\hat\lambda}{V} \sum\limits_{p,q,r,s} \delta_{p+r,q+s} \tilde\Phi_p^{\dagger} \tilde\Phi_q \tilde\Phi_r^{\dagger} \tilde\Phi_s 
    &= V\left( \hat\lambda\left( m_{\Phi}^4 + s_{\Phi}^4 + 6 m_{\Phi}^2s_{\Phi}^2 \right) \right) \nonumber \\
    &+ \frac{1}{2} \sum\limits_{0\neq p\neq p_s} \tilde\Phi_p^{\dagger} \left[ 8 \hat\lambda \left( m_{\Phi}^2 + s_{\Phi}^2 ) \right) \right]   \tilde\Phi_p \nonumber \\ 
    &+ \frac{\hat\lambda}{V} \widetilde{\sum\limits_{p,q,r,s}}\delta_{p+r,q+s} \tilde\Phi_p^{\dagger} \tilde\Phi_q \tilde\Phi_r^{\dagger} \tilde\Phi_s.
\end{align}
Here, the tilde above the some means, that \textit{none} of the occuring momenta is $0$~and/or~$p_s$.

Then, the total bosonic Gaussian contribution is:
\begin{equation}\label{eq:CEP_alternative_bosonic_Gaussian_contribution}
 S_{B,0} = \frac{1}{2}  \sum\limits_{0\neq p \neq p_s}  \tilde\Phi_p^{\dagger}  
              \left[ 2 - 4 \hat\lambda N_f + 8 \hat\lambda \left( m_{\Phi}^2 + s_{\Phi}^2  \right) - 4 \kappa \sum_{\mu} \cos(p_{\mu}) \right]  \tilde\Phi_p.
\end{equation}

If one then expands the interaction part to zeroth order, the calculation for the CEP looks like the following:
\begin{align}\label{eq:deriv_of_bosDet_step1}
 e^{-V\cdot U(m_{\Phi}, s_{\Phi})} &= 
           e^{-V\left(
             -8 \kappa \left( m_{\Phi}^2 - s_{\Phi}^2 \right)  +  m_{\Phi}^2 + s_{\Phi}^2 
             + \hat\lambda \left( m_{\Phi}^4 + s_{\Phi}^4 + 6 m_{\Phi}^2s_{\Phi}^2 - 2 N_f \left(m_{\Phi}^2 + s_{\Phi}^2\right) \right) \right)} 
             \nonumber \\
             &\times e^{-N_f \log \det \mathcal{M}[\Phi^g]} 
             \nonumber \\
             &\times \int \left[\prod\limits_{0 \neq p \neq p_s}  d \tilde \Phi_p \right] 
              e^{ \frac{1}{2}  \sum\limits_{0\neq p \neq p_s}  \tilde\Phi_p^{\dagger}  
              \left[ 2 - 4 \hat\lambda N_f + 8 \hat\lambda \left( m_{\Phi}^2 + s_{\Phi}^2  \right) - 4 \kappa \sum_{\mu} \cos(p_{\mu}) \right]  \tilde\Phi_p}, 
          \\ \label{eq:deriv_of_bosDet_step2}
          &= e^{-V U^{\text{tree}}(m_{\Phi}, s_{\Phi})} \times e^{-N_f \log \det \mathcal{M}[\Phi^g]} 
          \nonumber \\
          & \times \sqrt{\frac{V}{\prod\limits_{0 \neq p \neq p_s} 
             \left(  2 - 4 \hat\lambda N_f + 8 \hat\lambda \left( m_{\Phi}^2 + s_{\Phi}^2  \right) - 4 \kappa \sum_{\mu} \cos(p_{\mu}) \right) }} ,
             \\\label{eq:deriv_of_bosDet_step3}
          &=  e^{-V U^{\text{tree}}(m_{\Phi}, s_{\Phi})} \times e^{-N_f \log \det \mathcal{M}[\Phi^g]} 
            \nonumber \\
          & \times e^{\sum\limits_{0 \neq p \neq p_s} 
          \log\left( 2 - 4 \hat\lambda N_f + 8 \hat\lambda \left( m_{\Phi}^2 + s_{\Phi}^2  \right) - 4 \kappa \sum_{\mu} \cos(p_{\mu}) \right)}
            \times e^{\log V}.
\end{align}
\textbf{There might be an additional factor of 4 coming from the components of the bosonic field. this factor however can be absorbed in a constant that 
is neglected for the potential anyway. Is this correct?}
So finally the \textit{improved} zero-Order potential is given by:
\begin{align}\label{eq:improved_zeroOrderPot}
 U(m_{\Phi}, s_{\Phi}) &= -8 \kappa \left( m_{\Phi}^2 - s_{\Phi}^2 \right)   +   \left( m_{\Phi}^2 + s_{\Phi}^2 \right)
                         + \hat\lambda \left( m_{\Phi}^4 + s_{\Phi}^4 + 6 m_{\Phi}^2 s_{\Phi}^2 - 2 N_f \left(m_{\Phi}^2 + s_{\Phi}^2 \right) \right) 
                           \nonumber \\
                        & -\frac{2N_f}{V} \sum\limits_p \log
                                    \left[ \left( |\nu^+(p)| |\nu^+(\wp)|   +
                                    \hat y ^2 \left( m_{\Phi}^2 - s_{\Phi}^2 \right) |\gamma^+(p)| |\gamma^+(\wp)|\right)^2 
                                    \nonumber \right. \\ 
                        & \left. +  m_{\Phi}^2 \hat y^2 \left( |\gamma^+(p)| |\nu^+(\wp)|   -   |\nu^+(p)| |\gamma^+(\wp)| \right)^2\right]
                        \nonumber \\
                        & - \frac{1}{2V}\sum\limits_{0 \neq p \neq p_s} 
          \log\left( 2 - 4 \hat\lambda N_f + 8 \hat\lambda \left( m_{\Phi}^2 + s_{\Phi}^2  \right) - 4 \kappa \sum_{\mu} \cos(p_{\mu}) \right)
\end{align}
% 
% 
\subsubsection*{Alternative approach with a higher dimensional operator}

Here I will show what enters the bosonic determinant if a $\lambda_6 (\Phi^{\dagger}\Phi)^3$-term is included.
The decomposition with the naive approach looks like:
\begin{align}\label{eq:naive_decomposition_of_phi6_term}
 \frac{\hat\lambda_6}{V^2}\sum\limits_{p_1,\dots p6} \delta_{p_1 + p_3 + p_5, p_2 + p_4 + p_6} 
                     \tilde\Phi_{p_1}^{\dagger} \tilde\Phi_{p_2} \tilde\Phi_{p_3}^{\dagger} \tilde\Phi_{p_4} \tilde\Phi_{p_5}^{\dagger} \tilde\Phi_{p_6}
      &=  V \hat\lambda_6 \cdot \left( m_{\Phi}^6 + s_{\Phi}^6 + 15\cdot\left(m_{\Phi}^4 s_{\Phi}^2 + m_{\Phi}^2 s_{\Phi}^4 \right) \right) 
    \nonumber \\
      & + \frac{\hat\lambda_6}{V^2}\widehat{\sum\limits_{p_1,\dots p6}} \delta_{p_1 + p_3 + p_5, p_2 + p_4 + p_6} 
                     \tilde\Phi_{p_1}^{\dagger} \tilde\Phi_{p_2} \tilde\Phi_{p_3}^{\dagger} \tilde\Phi_{p_4} \tilde\Phi_{p_5}^{\dagger} \tilde\Phi_{p_6}
\end{align}
Here again one can take out a Gaussian contribution:
\begin{align}\label{eq:alternative_decomposition_of_phi6_term}
 dummy&=\frac{\hat\lambda_6}{V^2}\sum\limits_{p_1,\dots p6} \delta_{p_1 + p_3 + p_5, p_2 + p_4 + p_6} 
                     \tilde\Phi_{p_1}^{\dagger} \tilde\Phi_{p_2} \tilde\Phi_{p_3}^{\dagger} \tilde\Phi_{p_4} \tilde\Phi_{p_5}^{\dagger} \tilde\Phi_{p_6}
    \nonumber \\
      &=  V \hat\lambda_6 \left( m_{\Phi}^6 + s_{\Phi}^6 + 15\left(m_{\Phi}^4 s_{\Phi}^2 + m_{\Phi}^2 s_{\Phi}^4 \right) \right) 
    \nonumber \\
      & +  \frac{1}{2} \sum\limits_{0\neq p\neq p_s} \tilde\Phi_p^{\dagger} \left[ \hat\lambda_6\left( 18 \left( m_{\Phi}^4 + s_{\Phi}^4 \right) 
                       + 108\,  m_{\Phi}^2 s_{\Phi}^2\right)
                       \right]   \tilde\Phi_p
    \nonumber \\
      & + \frac{\hat\lambda_6}{V} \left( 9\left( m_{\Phi}^2 + s_{\Phi}^2 \right) \right)\widetilde{\sum\limits_{p_1,\dots p4}} \delta_{p_1 + p_3, p_2 + p_4 } 
                     \tilde\Phi_{p_1}^{\dagger} \tilde\Phi_{p_2} \tilde\Phi_{p_3}^{\dagger} \tilde\Phi_{p_4}
    \nonumber \\
      & + \frac{\hat\lambda_6}{V^2}\widetilde{\sum\limits_{p_1,\dots p6}} \delta_{p_1 + p_3 + p_5, p_2 + p_4 + p_6}
                     \tilde\Phi_{p_1}^{\dagger} \tilde\Phi_{p_2} \tilde\Phi_{p_3}^{\dagger} \tilde\Phi_{p_4} \tilde\Phi_{p_5}^{\dagger} \tilde\Phi_{p_6}.
\end{align}
As before, this gives a non-trivial contribution to the bosonic determinant. So finally, the CEP tree-level is given by:
\begin{align}\label{eq:improved_zeroOrderPot_withPhi6}
 U(m_{\Phi}, s_{\Phi}) &= -8 \kappa \left( m_{\Phi}^2 - s_{\Phi}^2 \right)   +   \left( m_{\Phi}^2 + s_{\Phi}^2 \right)
                         + \hat\lambda \left( m_{\Phi}^4 + s_{\Phi}^4 + 6 m_{\Phi}^2 s_{\Phi}^2 - 2 N_f \left(m_{\Phi}^2 + s_{\Phi}^2 \right) \right) 
         \nonumber \\
                        & + \hat\lambda_6 \left( m_{\Phi}^6 + s_{\Phi}^6 + 15 \left( m_{\Phi}^4 s_{\Phi}^2 + m_{\Phi}^2 s_{\Phi}^4 \right)   \right)
         \nonumber \\
                        & -\frac{2N_f}{V} \sum\limits_p \log
                                    \left[ \left( |\nu^+(p)| |\nu^+(\wp)|   +
                                    \hat y ^2 \left( m_{\Phi}^2 - s_{\Phi}^2 \right) |\gamma^+(p)| |\gamma^+(\wp)|\right)^2 
         \nonumber \right. \\
                        & \left. +  m_{\Phi}^2 \hat y^2 \left( |\gamma^+(p)| |\nu^+(\wp)|   -   |\nu^+(p)| |\gamma^+(\wp)| \right)^2\right]
         \nonumber \\
                        & - \frac{1}{2V}\sum\limits_{0 \neq p \neq p_s} 
          \log\left( 2 - 4 \hat\lambda N_f - 4 \kappa \sum_{\mu} \cos(p_{\mu}) + 8 \hat\lambda \left( m_{\Phi}^2 + s_{\Phi}^2  \right) 
                     +  18\, \hat\lambda_6\left( m_{\Phi}^4 + s_{\Phi}^4 + 6\, m_{\Phi}^2 s_{\Phi}^2 \right) \right)
\end{align}

The expression with the reparametrizations \eqref{eq:def_reparametrization_couplingsAndMagnetization_Nf} and \eqref{eq:lambda_6_reparametrized} for the rescaled potential
$\tilde{U}=U/N_f$ is straight foward. 
\begin{align}\label{eq:improved_zeroOrderPot_withPhi6_rescaled}
 \tilde U(\breve m_{\Phi}, \breve s_{\Phi}) &= -8 \kappa \left( \breve m_{\Phi}^2 - \breve s_{\Phi}^2 \right)   +   \left( \breve m_{\Phi}^2 + \breve s_{\Phi}^2 \right)
                         + \lambda_N \left( \breve m_{\Phi}^4 + \breve s_{\Phi}^4 + 6 \breve m_{\Phi}^2 \breve s_{\Phi}^2 
                         - 2  \left(\breve m_{\Phi}^2 + \breve s_{\Phi}^2 \right) \right) 
         \nonumber \\
                        & + \lambda_{6N} \left( \breve m_{\Phi}^6 + \breve s_{\Phi}^6 + 15 \left( \breve m_{\Phi}^4 \breve s_{\Phi}^2 
                          + \breve m_{\Phi}^2 \breve s_{\Phi}^4 \right)   \right)
         \nonumber \\
                        & -\frac{2}{V} \sum\limits_p \log
                                    \left[ \left( |\nu^+(p)| |\nu^+(\wp)|   +
                                    \hat y^2_N \left( \breve m_{\Phi}^2 - \breve s_{\Phi}^2 \right) |\gamma^+(p)| |\gamma^+(\wp)|\right)^2 
         \nonumber \right. \\
                        & \left. +  \breve m_{\Phi}^2 y^2_N \left( |\gamma^+(p)| |\nu^+(\wp)|   -   |\nu^+(p)| |\gamma^+(\wp)| \right)^2\right]
         \nonumber \\
                        & - \frac{1}{2VN_f}\sum\limits_{0 \neq p \neq p_s} 
          \log\left( 2 - 4 \lambda_N - 4 \kappa \sum_{\mu} \cos(p_{\mu}) + 8 \lambda_N \left( \breve m_{\Phi}^2 + \breve s_{\Phi}^2  \right) 
                     +  18\,\lambda_{6N} \left( \breve m_{\Phi}^4 + \breve s_{\Phi}^4 +6\, \breve  m_{\Phi}^2 \breve s_{\Phi}^2 \right) \right)
\end{align}






