\documentclass[a4paper,10pt,final]{article}
\usepackage[ansinew]{inputenc}
\usepackage{amssymb,amsmath,textcomp}
\usepackage{graphicx} 
\usepackage[margin=2cm]{geometry}
\usepackage{showkeys}
\usepackage[format=hang,small,singlelinecheck=true]{caption} 
\usepackage{subfig}	%2 grafiken nebeneinander	
\usepackage{multirow}	% mehrreihige Tabelleneintraege
\usepackage{slashed}   %for dirac-slash
\usepackage{simplewick}   %for wick contractios
\usepackage{url}
\usepackage{hyperref}
%%%to fight the ''Package hyperref Warning: bookmark level for unknown Chapter defaults to 0.''-warning
\makeatletter
\providecommand*{\toclevel@Chapter}{0}
\makeatother
%################################################################################
 \numberwithin{equation}{section} %nummerierung der Gleichungen mit section
%################################################################################
\title{Notes concerning the constrained effective potential}
\author{Attila Nagy}
\date{\today}


%%some newcommands%%
\newcommand{\Ls}{L_{_s}}
\newcommand{\Lt}{L_{_t}}
\newcommand{\kc}{\kappa_{c}}
\newcommand{\hl}{\hat{\lambda}}
\begin{document}
\section{General approach}
Here the general idea of the constrained effective potential (CEP) will be described. It is basically what is written in 
Philipp's thesis (chapter 4 and 6.3). The only real difference is, that in the thesis a general factor of $N_f$ drawn in front of the action
to consider the remaining part as independent from $N_f$. For that, some redefinitions of the coupling constants are performed. The reason to do that was, to investigate the phase structure of the model in the large $N_f$-limit. Further I will only quote the fermionic contributions, since I don't understand their derivation.

The general idea do consider a potential that only depends on an assumed ground state of the system. 
In the here considered model we consider the ground state
to consist of a zero momentum mode and a so called staggered mode:
\begin{equation}\label{eq:def_groundstate}
 \Phi^g_x=m_{\Phi} \cdot \hat \Phi_1 + s_{\Phi} \cdot e^{i p_s \cdot x}\cdot \hat \Phi_2 \qquad p_s=(\pi,\pi,\pi,\pi),
\end{equation}
with $\hat \Phi_{1/2}$ being constant vectors. Since the (relative) orientation of those vectors does not matter, 
one can assume them to be identical. The magnetization $m_{\Phi}$ and the staggered magnetization $s_{\Phi}$ will later be the order
parameters of the model.

The scalar field wil be considered in momentum space. As a reminder, Fourier transformation looks like:
\begin{equation}\label{eq:def_FourierTrafoScalar}
 \tilde \Phi_p = \frac{1}{\sqrt{V}}\sum\limits_x e^{-i\, p \cdot x} \Phi_x,\quad \Phi_x= \frac{1}{\sqrt{V}}\sum\limits_x e^{i\, p \cdot x} \tilde \Phi_p.
\end{equation}

The magnetizations are defined as follows:
\begin{equation}\label{eq:def_magnetizations}
 m_{\Phi}=\left| \frac{1}{V} \sum\limits_x \Phi_x \right| \rightarrow \tilde\Phi_0 = \sqrt{V}m_{\Phi},\quad
 s_{\Phi}=\left| \frac{1}{V} \sum\limits_x e^{i\, p_s \cdot x}\Phi_x \right| \rightarrow \tilde\Phi_{p_s} = \sqrt{V}s_{\Phi}.
\end{equation}

With that, the CEP $U(m,s)$ is defined via:
\begin{equation}\label{eq:def_CEP}
 V\cdot U(m,s) = -\log \left(\int D\Psi D\bar\Psi \left. \left[\prod\limits_{0 \neq p \neq p_s}  d \tilde \Phi_p \right]
          e^{-S[\Psi,\bar\Psi,\Phi]} 
          \right|_{ \begin{array}{l} \scriptscriptstyle \tilde \Phi_0=\sqrt{V} m_{\Phi} \\ 
                                     \scriptscriptstyle \tilde \Phi_{p_s}=\sqrt{V}s_{\Phi} \end{array}}  \right),
\end{equation}
with the action being composed of a fermionic ($S_F$) and a purely bosonic part ($S_B$).
The bosonic action can be written as follows:
\begin{equation}\label{eq:def_bosonicActionInLatticeNotation}
 S_B[\Phi] = -\kappa \sum\limits_{x,\mu} \Phi_x^{\dagger} \left[\Phi_{x+\hat\mu} + \Phi_{x-\hat\mu} \right] 
              + \sum\limits_x \Phi_x^{\dagger} \Phi_x 
              + \hat{\lambda} \sum\limits_x \left( \Phi_x^{\dagger} \Phi_x - N_f\right)^2 ,
\end{equation}
and the Fourier transformation of the individual parts are:
\begin{align}
 \sum\limits_{x} \Phi_x^{\dagger} \Phi_x &= 
                     \frac{1}{V}  \sum\limits_{x}  \sum\limits_{p,q}  \left( e^{-i\, p \cdot x}  \tilde\Phi_p^{\dagger} \right)
                     \left( e^{i\, q \cdot x}  \tilde\Phi_{q} \right), \nonumber \\
                  &= \frac{1}{V}  \sum\limits_{p,q,}  
                     \underbrace{ \sum\limits_{x}  e^{ -i (p - q) \cdot x} }_{V \cdot \delta_{p,q}}  \tilde\Phi_p^{\dagger} \tilde\Phi_q ,
                     \nonumber \\
                  &= \sum\limits_{p}  \tilde\Phi_p^{\dagger} \tilde\Phi_p 
                      \label{eq:FourierTrafoPhiSquared},
\end{align}
% 
\begin{align}
 \sum\limits_{x,\mu} \Phi_x^{\dagger} \left[\Phi_{x+\hat\mu} + \Phi_{x-\hat\mu} \right] &= 
                     \frac{1}{V}  \sum\limits_{x,\mu}  \sum\limits_{p,q}   e^{-i\, p \cdot x}  \tilde\Phi_p^{\dagger} 
                     \left[  e^{i\, q \cdot (x+\hat\mu)}  \tilde\Phi_{q}   +   e^{i\, q \cdot (x-\hat\mu)}  \tilde\Phi_{q}  \right], \nonumber \\
                  &= \frac{1}{V}  \sum\limits_{p,q,\mu}  
                     \sum\limits_{x}  e^{ -i (p - q) \cdot x}  \tilde\Phi_p^{\dagger} \tilde\Phi_q 
                     \underbrace{ \left(  e^{i\, q_{\mu}}  +  e^{-i\, q_{\mu}} \right)  }_{ 2\cdot \cos(q_{\mu}) }, \nonumber \\
                  &= \sum\limits_{p}  \tilde\Phi_p^{\dagger} \left[  2 \cdot \sum\limits_{\mu} \cos(p_{\mu})  \right]  \tilde\Phi_p 
                      \label{eq:FourierTrafoKappaTerm},
\end{align}
% 
\begin{align}
 \sum\limits_{x} \left( \Phi_x^{\dagger} \Phi_x \right)^2 &= 
                     \frac{1}{V^2}  \sum\limits_{x}  \sum\limits_{p,q,r,s}  
                     \left( e^{-i\, p \cdot x}  \tilde\Phi_p^{\dagger} \right) \left( e^{i\, q \cdot x}  \tilde\Phi_{q} \right)
                     \left( e^{-i\, r \cdot x}  \tilde\Phi_r^{\dagger} \right) \left( e^{i\, s \cdot x}  \tilde\Phi_{s} \right), \nonumber \\
                  &= \frac{1}{V^2}  \sum\limits_{p,q,r,s}  
                     \sum\limits_{x}  e^{ -i (p + r - q - s) \cdot x }  \tilde\Phi_p^{\dagger} \tilde\Phi_q \tilde\Phi_r^{\dagger} \tilde\Phi_s,
                     \nonumber \\
                  &= \frac{1}{V}\sum\limits_{p,q,r,s} \delta_{p+r,q+s} \tilde\Phi_p^{\dagger} \tilde\Phi_q \tilde\Phi_r^{\dagger} \tilde\Phi_s 
                      \label{eq:FourierTrafoPhiToTheFour}.
\end{align}
%
Now, in momentum space, the bosonic action can be written as:
\begin{align}
 S_B[\Phi] &= -\kappa  \sum\limits_{x,\mu}  \Phi_x^{\dagger}  \left[ \Phi_{x+\hat\mu} + \Phi_{x-\hat\mu} \right] 
              + \sum\limits_x \Phi_x^{\dagger} \Phi_x 
              + \hat{\lambda}  \sum\limits_x \left( \Phi_x^{\dagger} \Phi_x - N_f\right)^2, \nonumber \\
           &=  \frac{1}{2}  \sum\limits_{p}  \tilde\Phi_p^{\dagger}  
              \left[ 2 - 4 \hat\lambda N_f - 4 \kappa \sum_{\mu} \cos(p_{\mu}) \right]  \tilde\Phi_p
              + \frac{\hat\lambda}{V} \sum\limits_{p,q,r,s} \delta_{p+r,q+s} \tilde\Phi_p^{\dagger} \tilde\Phi_q \tilde\Phi_r^{\dagger} \tilde\Phi_s .
\end{align}
A constant term was ignored here, since we are only interested in terms depending on $m$ and $s$.
%
% 

If one could compute the CEP, one obtains some information: First of all, the absolute minimum of the CEP will be the ground state and if the 
minimum occures at zero or non-zero (staggered) magnetization determines the phase the system is in. Further, any observable that only depends on 
the $\tilde \Phi_0$ and/or $\Phi_{p_s}$ can be computed from a two-dimensional integral:
\begin{equation}\label{eq:obsFromCEP}
 \left< \mathcal{O}(\tilde \Phi_0,\tilde \Phi_{p_s}) \right > = 
      \mathcal{Z}^{-1} \int d \tilde \Phi_0\, d \tilde \Phi_{p_s} 
      \left( \mathcal{O}(\tilde \Phi_0,\tilde \Phi_{p_s})\, e^{-V\cdot U(\tilde \Phi_0,\tilde \Phi_{p_s})} \right)
\end{equation}

%
\clearpage
%
\section{Expansion of the action}
To compute the CEP perturbatively, one decomposes the action into a part, that only depends on the assumed ground state ($S_{B,\Phi^g}$), 
a gaussian part ($S_{B,0}$) where the fields only appear quadratically and an interaction part ($S_I$) that consiste of the rest. The treelevel part 
part can be computed directly \footnote{The computation of the fermionic contribution is somehow complicated, but in principle 
free of any perturbative arguments \textbf{check that}}.  The gaussian part can be integrated out. It gives a determinant and determines the propagators 
of the bosonic field. The determinant will be ignored in the following, since for what we need it does not matter, since it is indeppendent of $m$ and $s$.
The decomposition of the bosonic part is:
\begin{align}
 S_{B,\Phi^g} &= V \left( -8 \kappa \left( m_{\Phi}^2 - s_{\Phi}^2 \right)  +  m_{\Phi}^2 + s_{\Phi}^2 
             + \hat\lambda \left( m_{\Phi}^4 + s_{\Phi}^4 + 6 m_{\Phi}^2s_{\Phi}^2 - 2 N_f \left(m_{\Phi}^2 + s_{\Phi}^2\right) \right) \right) ,
             \label{eq:BosonicTreelevelCEP} \\
 S_{B,0}      &= \frac{1}{2}  \sum\limits_{0\neq p \neq p_s}  \tilde\Phi_p^{\dagger}  
              \left[ 2 - 4 \hat\lambda N_f - 4 \kappa \sum_{\mu} \cos(p_{\mu}) \right]  \tilde\Phi_p ,
              \label{eq:BosonicGaussianCEP} \\
 S_{B,I}        &= \frac{\hat\lambda}{V} \widehat{\sum\limits_{p,q,r,s}}\delta_{p+r,q+s} \tilde\Phi_p^{\dagger} \tilde\Phi_q \tilde\Phi_r^{\dagger} \tilde\Phi_s,
              \label{eq:BosonicInteractionCEP}
\end{align}
with $\hat\sum$ meaning, that at leas one of the summed momenta is not in $\{0,p_s\}$. For the prefactor ($6 m_{\Phi}^2s_{\Phi}^2$) in the treelevel 
expression one has to take into account, that the $\delta_{p,q}$ has to respect the periodicity, meaning that $p$ and $q$ are equal up to a 
difference of multiples of $2\pi$.

From $S_0$ on gets the contraction of the scala field:
\begin{equation}\label{eq:ContractionOfBosonicField}
 \contraction{}{\Phi}{{}^\dagger_p}{\Phi{}}  \Phi^\dagger_p \Phi_q = \frac{4 \cdot \delta_{p,q}}{2 - 4 \hat\lambda N_f - 4 \kappa \sum_{\mu} \cos(p_{\mu}) }.
\end{equation}
The factor 4 in the numerator comes from the fact, that this contraction is performed on every component of the scalar field. \textbf{Since this is a crucial statement, find someone to check this!}

For the exponential of the CEP we can write now (including the fermionic Part, that will contribute):

\begin{equation} \label{eq:expandedCEP}
 e^{-V\cdot U(m,s)} = e^{-S_{B,\Phi^g}[\Phi^g]} e^{-N_f \log \det \mathcal{M}[\Phi^g]} 
                      \int \left. \left[\prod\limits_{0 \neq p \neq p_s}  d \tilde \Phi_p \right]   
                      e^{-S_{B,I}[\Phi] +  S_{B,0}[\Phi]}
                       \right|_{ \begin{array}{l} \scriptscriptstyle \tilde \Phi_0=\sqrt{V} m_{\Phi} \\ 
                                     \scriptscriptstyle \tilde \Phi_{p_s}=\sqrt{V}s_{\Phi} \end{array}},
\end{equation}
If one expands the exponential from $S_{B,I}$ to first order, the integral in \eqref{eq:expandedCEP} we end up with:
\begin{align}\label{eq:CEP_firstOrder}
 \int \left[ \prod\limits_{0 \neq p \neq p_s}  d \tilde \Phi_p \right]   
                      e^{-S_{B,I}[\Phi] +  S_{B,0}[\Phi]} =& \int \left[ \prod\limits_{0 \neq p \neq p_s}  d \tilde \Phi_p \right] 
                      \left( 1 - \frac{\hat\lambda}{V} \widehat{\sum\limits_{p,q,r,s}}\delta_{p+r,q+s} 
                      \tilde\Phi_p^{\dagger} \tilde\Phi_q \tilde\Phi_r^{\dagger} \tilde\Phi_s \right)e^{-S_B,0} \nonumber \\
                      =& \sqrt{\frac{V}{\det \mathcal{B}}} \cdot \left(1 - \frac{\hat\lambda}{V} \left( C_{p,q} + C_{m,p} + C_{s,p} \right) \right) \nonumber \\
                      =& \sqrt{\frac{V}{\det \mathcal{B}}} \cdot e^{-\frac{\hat\lambda}{V} \left( C_{p,q} + C_{m,p} + C_{s,p} \right)}
\end{align}
In the above equations, $\det\mathcal{B}$ is the bosonic determinant. The matrix $\mathcal{B}$ is given by:
\begin{equation}\label{eq:def_BosonicDeterminant}
\mathcal{B}^{i,j}(p,q)=\delta_{i,j}\delta_{p,q} \left( 2 - 4 \hat\lambda N_f - 4 \kappa \sum\limits_{\mu} \cos(p_{\mu}) \right)
\end{equation}
\textbf{Check, if determinant occurs in the correct power!}
The indices $i$ and $j$ label the components of the scalar field. It is independet of $m_{\Phi}/s_{\Phi}$ and will, 
when the log of the r.h.s. of \eqref{eq:CEP_firstOrder} is taken, only contributa a constant and can be neglected.
The $C$ come from the different possibilities to contract the four fields. 
The first one, $C_{p,q}$ comes from a full contraction:
\begin{equation}\label{eq:CEP_firstOrderLambda_VacuumBubble}
 \contraction{}{\tilde\Phi}{{}_p^{\dagger}}{\tilde\Phi} \tilde\Phi_p^{\dagger} \tilde\Phi_q
  \contraction{}{\tilde\Phi}{{}_r^{\dagger}}{\tilde\Phi} \tilde\Phi_r^{\dagger} \tilde\Phi_s = 
    \frac{ \delta_{p,q}}{2 - 4 \hat\lambda N_f - 4 \kappa \sum_{\mu} \cos(p_{\mu}) } \cdot 
    \frac{ \delta_{r,s}}{2 - 4 \hat\lambda N_f - 4 \kappa \sum_{\mu} \cos(r_{\mu}) },
\end{equation}
and gives some kind of vacuum bubble which is independent of zero or staggered mode 
and will therefore be neglected here.
The other two contributions in the exponantial origin from those terms in the sum, where some (but not all) occuring momenta are either $0$ or $p_s$:
\begin{align}\label{eq:CEP_firstOrderLambda_zeroModeContribution}
 C_{p,m}=& \widehat{\sum\limits_{p,q}} \left( 
             \contraction{}{\Phi}{{}_p^{\dagger}}{\Phi}                              \Phi_p^{\dagger} \Phi_q \Phi_0^{\dagger} \Phi_0
           + \contraction{}{\Phi}{{}_p^{\dagger} \Phi_0 \Phi_0^{\dagger}}{\Phi}      \Phi_p^{\dagger} \Phi_0 \Phi_0^{\dagger} \Phi_q
           + \contraction{\Phi_0^{\dagger} }{\Phi}{{}_p}{\Phi}                       \Phi_0^{\dagger} \Phi_p \Phi_q^{\dagger} \Phi_0
           + \contraction{\Phi_0^{\dagger} \Phi_0 }{\Phi}{{}_p^{\dagger}}{\Phi}      \Phi_0^{\dagger} \Phi_0 \Phi_p^{\dagger} \Phi_q
            \right) \nonumber \\
        =& 16 \cdot V \cdot m_{\Phi}^2 \sum\limits_{0\neq p \neq p_s} \frac{1}{2 - 4 \hat\lambda N_f - 4 \kappa \sum_{\mu} \cos(p_{\mu})} \\
  \label{eq:CEP_firstOrderLambda_staggeredModeContribution}
 C_{p,s}=& 16 \cdot V \cdot s_{\Phi}^2 \sum\limits_{0\neq p \neq p_s} \frac{1}{2 - 4 \hat\lambda N_f - 4 \kappa \sum_{\mu} \cos(p_{\mu})}.
\end{align}
For later use, the propagator sum will be defined:
\begin{equation}\label{eq:bosonicPropagatorSum_CEP}
 P_B \equiv \frac{1}{V}\sum\limits_{0\neq p\neq p_s} \frac{1}{2 - 4 \hat\lambda N_f - 4 \kappa \sum_{\mu} \cos(p_{\mu})}.
\end{equation}



%
\clearpage
%
\section{Determination of the phase structure}
To determine the phase structure of the model, i.e.\ determine the phase the system is in depending on the choice of parameters, 
one has to find out, what the minimum of the CEP is w.r.t.\ $m_{\Phi}$ and $s_{\Phi}$. The simplest ansatz, would be the tree-level (again with fermions):
\begin{align}\label{eq:CEP_treelevel}
 U(m_{\Phi}, s_{\Phi}) =& -8 \kappa \left( m_{\Phi}^2 - s_{\Phi}^2 \right)   +   \left( m_{\Phi}^2 + s_{\Phi}^2 \right)
                         + \hat\lambda \left( m_{\Phi}^4 + s_{\Phi}^4 + 6 m_{\Phi}^2 s_{\Phi}^2 - 2 N_f \left(m_{\Phi}^2 + s_{\Phi}^2 \right) \right) 
                           \nonumber \\
                        & -\frac{2N_f}{V} \log \sum\limits_p
                                    \left[ \left( |\nu^+(p)| |\nu^+(\wp)|   +
                                    \hat y ^2 \left( m_{\Phi}^2 - s_{\Phi}^2 \right) |\gamma^+(p)| |\gamma^+(\wp)|\right)^2 
                                    \nonumber \right. \\ 
                        & \left. +  m_{\Phi}^2 \hat y^2 \left( |\gamma^+(p)| |\nu^+(\wp)|   -   |\nu^+(p)| |\gamma^+(\wp)| \right)^2\right]
\end{align}
Here $\nu^+(p)$ labels the eigenvalue of the overlap operator\footnote{In the definition of this eigenvalue and how it is implemeted in the 
code, there is a difference in the prefactor of $r$, the Wilson parameter. I think, its origin may lie in an ambiguity of
the definition of the Wilson parameter. What is written hiere, is how it is in the code and how I implemented it}. 
It expression and the other abreviations used are ($a=1$):
\begin{equation}
 \label{eq:eigenvalue_overlapOperator_and_gamma}
 \nu^{\pm}(p) = \rho \left( 1 + 
                      \frac{ \pm i \sqrt{{\tilde p} ^2} + r  {\hat p}^2 - \rho}
                           {\sqrt{ {\tilde p} ^2 + \left( r  {\hat p}^2 - \rho\right)^2}}\right),\qquad
 \gamma^{\pm}(p) = 1- \frac{1}{2 \rho}\nu^{\pm}(p),
 \end{equation}
 \begin{equation}
 \label{eq:def_momentumAbreviations}
 {\hat p}^2 = 4 \sum\limits_{\mu} \sin^2\left(\frac{p_{\mu}}{2}\right),\qquad
 {\tilde p}^2 = \sum\limits_{\mu} \sin^2\left(p_{\mu}\right),\qquad
 \wp = p + p_s.
\end{equation}

This is however just a very crude approximation. In Philipp's thesis, he draw a common factor of $N_f$ in front of the action and redefined the couplings 
and the magnetizations according to:
\begin{equation}\label{eq:def_reparametrization_couplingsAndMagnetization_Nf}
 \hat y = \frac{y_N}{\sqrt{N_f}},\qquad
 \hat \lambda = \frac{\lambda_N}{N_f},\qquad
 \breve{m}/_{\Phi} = \sqrt{N_f} m_{\Phi},\qquad
 \breve{s}/_{\Phi} = \sqrt{N_f} s_{\Phi}.
\end{equation}
The reason for that is, if you also rescale the scalar field according to $\breve{\Phi}=\sqrt{N_f}\Phi$, the action becomes independet of $N_f$. 
When keeping now $\lambda_N$ and $y_N$ constant when performing the limit $N_f \rightarrow \infty$, everything but the treelevel get's suppressed by 
powers of $\frac{1}{N_f}$.
To reproduce the expression used by Philipp, also draw a factor in front of $U$:
\begin{align}\label{eq:CEP_treelevel_reparametrized}
 \tilde U(\breve{m}_{\Phi},\breve{s}_{\Phi}) =& \frac{U}{N_f} \nonumber \\
            =& -8 \kappa \left( \breve{m}_{\Phi}^2 - \breve{s}_{\Phi}^2 \right)   +   \left( \breve{m}_{\Phi}^2 + \breve{s}_{\Phi}^2 \right)
                         + \lambda_N \left( \breve{m}_{\Phi}^4 + s_{\Phi}^4 + 6 \breve{m}_{\Phi}^2 \breve{s}_{\Phi}^2
                         - 2 \left(\breve{m}_{\Phi}^2 + s_{\Phi}^2 \right) \right) 
                           \nonumber \\
                        & -\frac{2}{V} \log \sum\limits_p
                                    \left[ \left( |\nu^+(p)| |\nu^+(\wp)|   +
                                    \hat y ^2 \left( m_{\Phi}^2 - \breve{s}_{\Phi}^2 \right) |\gamma^+(p)| |\gamma^+(\wp)|\right)^2 
                                    \nonumber \right. \\ 
                        & \left. +  \breve{m}_{\Phi}^2 y_N^2 \left( |\gamma^+(p)| |\nu^+(\wp)|   -   |\nu^+(p)| |\gamma^+(\wp)| \right)^2\right].
\end{align}

The addition of the first order term in $\hat \lambda$ is straight foward:
\begin{equation}\label{eq:CEP_oneLoopAddition}
 U(m_{\Phi}, s_{\Phi}) \rightarrow U(m_{\Phi}, s_{\Phi}) + 16\, \hat \lambda\, P_B, \qquad 
 \tilde U(\breve{m}_{\Phi},\breve{s}_{\Phi}) \rightarrow  \tilde U(\breve{m}_{\Phi},\breve{s}_{\Phi}) + 16 \,\frac{\lambda_N}{N_f}\, P_B.
\end{equation}





\subsection*{Limits of the CEP and problems with first order in $\lambda$}
The whole approach of integrating out the Gaussian contribution is only valid, if the bosonic determinant~\eqref{eq:def_BosonicDeterminant}
is positive definit. This limits the range of values for $\hat\lambda$ and $\kappa$:
\begin{equation}\label{eq:limits_for_CEP_from_matrix}
 0 \stackrel{!}{<} 2 - 4 N_f - 4 \kappa \sum\limits_{\mu} \sin(p_{\mu}) \quad \Rightarrow \quad
 \hat\lambda < \frac{1}{2 N_f} ,\qquad
 |\kappa| < \frac{1 - 2 N_f \hat \lambda}{8}.
\end{equation}
While this might only be a somehow estitical problem for the tree-level, it gets serious when including the the first order in 
$\lambda$ (\eqref{eq:CEP_firstOrderLambda_zeroModeContribution} and \eqref{eq:CEP_firstOrderLambda_zeroModeContribution}), since the denominator
becomes negative or (close to) zero, spoiling everything.
This is extremely unpleasant, since the first order in $\lambda$ has a huge effect. To see this, remove the fermions and assume, you are in a phase, where 
there should be either the symmetric or the ferromagnetic phase, i.e.\ $s_{\Phi}=0$. Then the CEP reduces to:
\begin{equation}\label{eq:simple_CEP_reduction}
 U(m_{\Phi}) = \Bigl( \underbrace{1 - 8\kappa - 2 N_f \hat \lambda}_{\mathcal{O}(0\dots 2)} 
             + \underbrace{ 16\, \hat\lambda\, P_B}_{\mathcal{O}(10\hat\lambda)} \Bigr)\cdot m_{\Phi}^2 + \hat \lambda m_{\Phi}^4.
\end{equation}
For this potential, to have a minimum at $m_{\Phi}\neq 0$, the coefficient of the quadratic term must be negative. Without the first order term, this happens 
when $\kappa$ violates the condition~\eqref{eq:limits_for_CEP_from_matrix}. With fermions turned on, they help, that a non-zero minimum 
is found without violating this condition. However, the effect of the fermions is only small. If one now compares the order of magnitude, the first order term in 
$\hat\lambda$ give a very strong contribution which cannot be compensated by the fermions anymore (with resonable couplings at least).



%
\clearpage
%
\section{Inclusion of a higher dimensional operator}
Here the addition of a higher dimensional operator should be discussed. The new bosonic action in the lattice notation is then:
\begin{equation}\label{eq:bosonicLatticeActionWithLambda_6}
 S_B[\Phi]  = -\kappa \sum\limits_{x,\mu} \Phi_x^{\dagger} \left[\Phi_{x+\hat\mu} + \Phi_{x-\hat\mu} \right] 
              + \sum\limits_x \Phi_x^{\dagger} \Phi_x 
              + \hat{\lambda} \sum\limits_x \left( \Phi_x^{\dagger} \Phi_x - N_f\right)^2 + \hat \lambda_6 \sum\limits_x \left( \Phi_x^{\dagger} \Phi_x \right)^3
\end{equation}
With that, the CEP changes. With the given notation however, only the tree level part changes, since the new parameter does not enter the gaussion contribution.
The Fourier transorm of the new term looks like:
\begin{align}
 \sum\limits_{x} \left( \Phi_x^{\dagger} \Phi_x \right)^3 &= 
                     \frac{1}{V^3}  \sum\limits_{x}  \sum\limits_{p_1,\dots p6}  
                     \left( e^{-i\, p_1 \cdot x}  \tilde\Phi_{p_1}^{\dagger} \right) \left( e^{i\, p_2 \cdot x}  \tilde\Phi_{p_2} \right)
                     \left( e^{-i\, p_3 \cdot x}  \tilde\Phi_{p_3}^{\dagger} \right) \left( e^{i\, p_4 \cdot x}  \tilde\Phi_{p_4} \right)
                     \left( e^{-i\, p_5 \cdot x}  \tilde\Phi_{p_5}^{\dagger} \right) \left( e^{i\, p_6 \cdot x}  \tilde\Phi_{p_6} \right), \nonumber \\
                  &= \frac{1}{V^3}  \sum\limits_{p_1,\dots p6} 
                     \sum\limits_{x}  e^{ -i (p_1 + p_3 + p_5 - p_2 - p_4 - p_6) \cdot x }  
                     \tilde\Phi_{p_1}^{\dagger} \tilde\Phi_{p_2} \tilde\Phi_{p_3}^{\dagger} \tilde\Phi_{p_4} \tilde\Phi_{p_5}^{\dagger} \tilde\Phi_{p_6},
                     \nonumber \\
                  &= \frac{1}{V^2}\sum\limits_{p_1,\dots p6} \delta_{p_1 + p_3 + p_5, p_2 + p_4 + p_6} 
                     \tilde\Phi_{p_1}^{\dagger} \tilde\Phi_{p_2} \tilde\Phi_{p_3}^{\dagger} \tilde\Phi_{p_4} \tilde\Phi_{p_5}^{\dagger} \tilde\Phi_{p_6}
                      \label{eq:FourierTrafoPhiToTheSix}.
\end{align}
The contribution do the tree-level of the CEP is then:
\begin{multline}\label{eq:CEP_tree_level_Contribution_from_lambda_6}
 U^{\text{tree}}(m_{\Phi},s_{\Phi}) = -8 \kappa \left( m_{\Phi}^2 - s_{\Phi}^2 \right)   +   \left( m_{\Phi}^2 + s_{\Phi}^2 \right) 
                         + \hat\lambda \left( m_{\Phi}^4 + s_{\Phi}^4 + 6 m_{\Phi}^2 s_{\Phi}^2 - 2 N_f \left(m_{\Phi}^2 + s_{\Phi}^2 \right) \right) \\
                         + \hat\lambda_6 \left( m_{\Phi}^6 + s_{\Phi}^6 + 15 m_{\Phi}^4 s_{\Phi}^2 + 15 m_{\Phi}^2 s_{\Phi}^4\right).
\end{multline}
\textbf{let this check!}
If one changes to the notation in Philipp's thesis, with the common factor $N_f$ drawn out, one has to reparametrize $\hat\lambda_6$:
\begin{equation}\label{eq:lambda_6_reparametrized}
 \hat \lambda_6 = \frac{\lambda_{6N}}{N_f^2},
\end{equation}
leading to an equivalent contribution to $\tilde U(\breve{m}_{\Phi},\breve{s}_{\Phi})$.

\subsection*{Higher diemensional operators and 1st order in $\hat\lambda_6$}

If one wants to compute the first order contribution in $\hat\lambda_6$, analogous to~\eqref{eq:CEP_oneLoopAddition}, one must consider all possible contractions.
As for the first order in $\hat \lambda$, the full contraction of all six fields does not contribute, since it is independent of $m_{\Phi}$ and $s_{\Phi}$. Since texing the 
contractions is annoying, the combinatorics can be found in tab.~\ref{tab:combFactors_1stOrder_lambda6}. \textbf{In addition to thoase combinatoric
factors a factor of 4 has to be taken into account for each contraction of momenta coming from the nominator in eq.~\eqref{eq:ContractionOfBosonicField}.}
With this, the contribution to the potential is:
\begin{align}\label{eq:CEP_1stOrderContribution_lambda6}
 U(m_{\Phi}, s_{\Phi}) & \rightarrow 
                          U(m_{\Phi}, s_{\Phi})  +  \hat \lambda_6\, \left[ 36\, \left(m_{\Phi}^4 + s_{\Phi}^4 + 6 m_{\Phi}^2 s_{\Phi}^2\right) P_B 
                          + 576 \left(m_{\Phi}^2 + s_{\Phi}^2 \right) P_B^2 \right],
        \\ \label{eq:CEP_1stOrderContribution_lambda6_rescaled}
 \tilde U(\breve{m}_{\Phi},\breve{s}_{\Phi}) & \rightarrow  
       \tilde U(\breve{m}_{\Phi},\breve{s}_{\Phi}) 
                          + \hat \lambda_6\, \left[ \frac{36}{N_f} \left(\breve m_{\Phi}^4 + \breve s_{\Phi}^4 + 6 \breve m_{\Phi}^2 \breve s_{\Phi}^2\right) P_B 
                          + \frac{576}{N_f^2} \left(\breve m_{\Phi}^2 + \breve s_{\Phi}^2 \right) P_B^2 \right],
\end{align}

\begin{table}[htb]\centering{
%   \rowcolors{1}{white}{lightgray}
  \begin{tabular}{| c | c | c | c || c | c | c | c |}
		\hline
		coefficient              & $p_1$ & $p_3$ & $p_5$ & $p_2$ & $p_4$ & $p_6$ & factor               \\ \hline
		$m_{\Phi}^2$             &  0    &  $p$  &  $q$  &   0   &  $p$  &  $q$  & $6\times6=36$        \\ \hline
		$s_{\Phi}^2$             & $p_s$ &  $p$  &  $q$  & $p_s$ &  $p$  &  $q$  & $6\times6=36$        \\ \hline
		$m_{\Phi}^4$             &   0   &  0    &  $p$  &   0   &   0   &  $p$  & $3\times3=9$         \\ \hline
		$s_{\Phi}^4$             & $p_s$ & $p_s$ &  $p$  & $p_s$ & $p_s$ &  $p$  & $3\times3=9$         \\ \hline
		$m_{\Phi}^2\,m_{\Phi}^2$ &   0   &  0    &  $p$  & $p_s$ & $p_s$ &  $p$  & $3\times3\times2=18$ \\ \hline
		$m_{\Phi}^2\,m_{\Phi}^2$ &   0   & $p_s$ &  $p$  &   0   & $p_s$ &  $p$  & $6\times6       =36$ \\ \hline
  \end{tabular}}
  \caption{Combinatoric factors for the inclusion of the first order contribution of $\hat\lambda_6$}\label{tab:combFactors_1stOrder_lambda6}
\end{table}



%
\clearpage
%
\section{Alternative expansion of the bosonic action}
\label{ch:alternativeExpansion}
Here another way to expand the action should be discussed. More accurate: More terms will contribute to the Gaussian part of the action $S_{B,0}$ 
which might improve the predictive power of the CEP without the ened to include the one loop term. What will be done is: Take from the interactive part those
term, that only have two of the bosonic fields being not the zero or staggered mode. With that, they can be considered as contribution to the 
Gaussioan part. The drawback from this is, that one has to include the bosonic determinant, since it then depends on $m_{\Phi}$ and $s_{\Phi}$. Further,
if one considers further orders in $\hat\lambda$, the propagator sums will be more complicated and also vacuum bubbles are not independent of $m_{\Phi}$ and $s_{\Phi}$
anymore.

Starting point is the expression for the $\hat\lambda \left( \Phi^{\dagger}\Phi \right)^2$ term. So far, the decomposition of this term looked like:
\begin{equation}\label{eq:naive_decomposition_of_phi4_term}
 \frac{\hat\lambda}{V} \sum\limits_{p,q,r,s} \delta_{p+r,q+s} \tilde\Phi_p^{\dagger} \tilde\Phi_q \tilde\Phi_r^{\dagger} \tilde\Phi_s 
    = V\left( \hat\lambda\left( m_{\Phi}^4 + s_{\Phi}^4 + 6 m_{\Phi}^2s_{\Phi}^2 \right) \right)
    + \frac{\hat\lambda}{V} \widehat{\sum\limits_{p,q,r,s}}\delta_{p+r,q+s} \tilde\Phi_p^{\dagger} \tilde\Phi_q \tilde\Phi_r^{\dagger} \tilde\Phi_s,
\end{equation}
with the hat above the sum indicating that not \textit{all} occuring momenta are eigther $0$~and/or~$p_s$. The terms in the sum, where two of the momenta 
are eigther $0$ or $p_s$ are then taken care of in the first order expansion of $\hat\lambda$.
Another possibility would be:
\begin{align}\label{eq:alternative_decomposition_of_phi4_term}
 \frac{\hat\lambda}{V} \sum\limits_{p,q,r,s} \delta_{p+r,q+s} \tilde\Phi_p^{\dagger} \tilde\Phi_q \tilde\Phi_r^{\dagger} \tilde\Phi_s 
    &= V\left( \hat\lambda\left( m_{\Phi}^4 + s_{\Phi}^4 + 6 m_{\Phi}^2s_{\Phi}^2 \right) \right) \nonumber \\
    &+ \frac{1}{2} \sum\limits_{0\neq p\neq p_s} \tilde\Phi_p^{\dagger} \left[ 8 \hat\lambda \left( m_{\Phi}^2 + s_{\Phi}^2 ) \right) \right]   \tilde\Phi_p \nonumber \\ 
    &+ \frac{\hat\lambda}{V} \widetilde{\sum\limits_{p,q,r,s}}\delta_{p+r,q+s} \tilde\Phi_p^{\dagger} \tilde\Phi_q \tilde\Phi_r^{\dagger} \tilde\Phi_s.
\end{align}
Here, the tilde above the some means, that \textit{none} of the occuring momenta is $0$~and/or~$p_s$.

Then, the total bosonic Gaussian contribution is:
\begin{equation}\label{eq:CEP_alternative_bosonic_Gaussian_contribution}
 S_{B,0} = \frac{1}{2}  \sum\limits_{0\neq p \neq p_s}  \tilde\Phi_p^{\dagger}  
              \left[ 2 - 4 \hat\lambda N_f + 8 \hat\lambda \left( m_{\Phi}^2 + s_{\Phi}^2  \right) - 4 \kappa \sum_{\mu} \cos(p_{\mu}) \right]  \tilde\Phi_p.
\end{equation}

If one then expands the interaction part to zeroth order, the calculation for the CEP looks like the following:
\begin{align}\label{eq:deriv_of_bosDet_step1}
 e^{-V\cdot U(m_{\Phi}, s_{\Phi})} &= 
           e^{-V\left(
             -8 \kappa \left( m_{\Phi}^2 - s_{\Phi}^2 \right)  +  m_{\Phi}^2 + s_{\Phi}^2 
             + \hat\lambda \left( m_{\Phi}^4 + s_{\Phi}^4 + 6 m_{\Phi}^2s_{\Phi}^2 - 2 N_f \left(m_{\Phi}^2 + s_{\Phi}^2\right) \right) \right)} 
             \nonumber \\
             &\times e^{-N_f \log \det \mathcal{M}[\Phi^g]} 
             \nonumber \\
             &\times \int \left[\prod\limits_{0 \neq p \neq p_s}  d \tilde \Phi_p \right] 
              e^{ \frac{1}{2}  \sum\limits_{0\neq p \neq p_s}  \tilde\Phi_p^{\dagger}  
              \left[ 2 - 4 \hat\lambda N_f + 8 \hat\lambda \left( m_{\Phi}^2 + s_{\Phi}^2  \right) - 4 \kappa \sum_{\mu} \cos(p_{\mu}) \right]  \tilde\Phi_p}, 
          \\ \label{eq:deriv_of_bosDet_step2}
          &= e^{-V U^{\text{tree}}(m_{\Phi}, s_{\Phi})} \times e^{-N_f \log \det \mathcal{M}[\Phi^g]} 
          \nonumber \\
          & \times \sqrt{\frac{V}{\prod\limits_{0 \neq p \neq p_s} 
             \left(  2 - 4 \hat\lambda N_f + 8 \hat\lambda \left( m_{\Phi}^2 + s_{\Phi}^2  \right) - 4 \kappa \sum_{\mu} \cos(p_{\mu}) \right) }} ,
             \\\label{eq:deriv_of_bosDet_step3}
          &=  e^{-V U^{\text{tree}}(m_{\Phi}, s_{\Phi})} \times e^{-N_f \log \det \mathcal{M}[\Phi^g]} 
            \nonumber \\
          & \times e^{\sum\limits_{0 \neq p \neq p_s} 
          \log\left( 2 - 4 \hat\lambda N_f + 8 \hat\lambda \left( m_{\Phi}^2 + s_{\Phi}^2  \right) - 4 \kappa \sum_{\mu} \cos(p_{\mu}) \right)}
            \times e^{\log V}.
\end{align}
\textbf{There might be an additional factor of 4 coming from the components of the bosonic field. this factor however can be absorbed in a constant that 
is neglected for the potential anyway. Is this correct?}
So finally the \textit{improved} zero-Order potential is given by:
\begin{align}\label{eq:improved_zeroOrderPot}
 U(m_{\Phi}, s_{\Phi}) &= -8 \kappa \left( m_{\Phi}^2 - s_{\Phi}^2 \right)   +   \left( m_{\Phi}^2 + s_{\Phi}^2 \right)
                         + \hat\lambda \left( m_{\Phi}^4 + s_{\Phi}^4 + 6 m_{\Phi}^2 s_{\Phi}^2 - 2 N_f \left(m_{\Phi}^2 + s_{\Phi}^2 \right) \right) 
                           \nonumber \\
                        & -\frac{2N_f}{V} \sum\limits_p \log
                                    \left[ \left( |\nu^+(p)| |\nu^+(\wp)|   +
                                    \hat y ^2 \left( m_{\Phi}^2 - s_{\Phi}^2 \right) |\gamma^+(p)| |\gamma^+(\wp)|\right)^2 
                                    \nonumber \right. \\ 
                        & \left. +  m_{\Phi}^2 \hat y^2 \left( |\gamma^+(p)| |\nu^+(\wp)|   -   |\nu^+(p)| |\gamma^+(\wp)| \right)^2\right]
                        \nonumber \\
                        & - \frac{1}{2V}\sum\limits_{0 \neq p \neq p_s} 
          \log\left( 2 - 4 \hat\lambda N_f + 8 \hat\lambda \left( m_{\Phi}^2 + s_{\Phi}^2  \right) - 4 \kappa \sum_{\mu} \cos(p_{\mu}) \right)
\end{align}
% 
% 
\subsubsection*{Alternative approach with a higher dimensional operator}

Here I will show what enters the bosonic determinant if a $\lambda_6 (\Phi^{\dagger}\Phi)^3$-term is included.
The decomposition with the naive approach looks like:
\begin{align}\label{eq:naive_decomposition_of_phi6_term}
 \frac{\hat\lambda_6}{V^2}\sum\limits_{p_1,\dots p6} \delta_{p_1 + p_3 + p_5, p_2 + p_4 + p_6} 
                     \tilde\Phi_{p_1}^{\dagger} \tilde\Phi_{p_2} \tilde\Phi_{p_3}^{\dagger} \tilde\Phi_{p_4} \tilde\Phi_{p_5}^{\dagger} \tilde\Phi_{p_6}
      &=  V \hat\lambda_6 \cdot \left( m_{\Phi}^6 + s_{\Phi}^6 + 15\cdot\left(m_{\Phi}^4 s_{\Phi}^2 + m_{\Phi}^2 s_{\Phi}^4 \right) \right) 
    \nonumber \\
      & + \frac{\hat\lambda_6}{V^2}\widehat{\sum\limits_{p_1,\dots p6}} \delta_{p_1 + p_3 + p_5, p_2 + p_4 + p_6} 
                     \tilde\Phi_{p_1}^{\dagger} \tilde\Phi_{p_2} \tilde\Phi_{p_3}^{\dagger} \tilde\Phi_{p_4} \tilde\Phi_{p_5}^{\dagger} \tilde\Phi_{p_6}
\end{align}
Here again one can take out a Gaussian contribution:
\begin{align}\label{eq:alternative_decomposition_of_phi6_term}
 dummy&=\frac{\hat\lambda_6}{V^2}\sum\limits_{p_1,\dots p6} \delta_{p_1 + p_3 + p_5, p_2 + p_4 + p_6} 
                     \tilde\Phi_{p_1}^{\dagger} \tilde\Phi_{p_2} \tilde\Phi_{p_3}^{\dagger} \tilde\Phi_{p_4} \tilde\Phi_{p_5}^{\dagger} \tilde\Phi_{p_6}
    \nonumber \\
      &=  V \hat\lambda_6 \left( m_{\Phi}^6 + s_{\Phi}^6 + 15\left(m_{\Phi}^4 s_{\Phi}^2 + m_{\Phi}^2 s_{\Phi}^4 \right) \right) 
    \nonumber \\
      & +  \frac{1}{2} \sum\limits_{0\neq p\neq p_s} \tilde\Phi_p^{\dagger} \left[ \hat\lambda_6\left( 18 \left( m_{\Phi}^4 + s_{\Phi}^4 \right) 
                       + 108\,  m_{\Phi}^2 s_{\Phi}^2\right)
                       \right]   \tilde\Phi_p
    \nonumber \\
      & + \frac{\hat\lambda_6}{V} \left( 9\left( m_{\Phi}^2 + s_{\Phi}^2 \right) \right)\widetilde{\sum\limits_{p_1,\dots p4}} \delta_{p_1 + p_3, p_2 + p_4 } 
                     \tilde\Phi_{p_1}^{\dagger} \tilde\Phi_{p_2} \tilde\Phi_{p_3}^{\dagger} \tilde\Phi_{p_4}
    \nonumber \\
      & + \frac{\hat\lambda_6}{V^2}\widetilde{\sum\limits_{p_1,\dots p6}} \delta_{p_1 + p_3 + p_5, p_2 + p_4 + p_6}
                     \tilde\Phi_{p_1}^{\dagger} \tilde\Phi_{p_2} \tilde\Phi_{p_3}^{\dagger} \tilde\Phi_{p_4} \tilde\Phi_{p_5}^{\dagger} \tilde\Phi_{p_6}.
\end{align}
As before, this gives a non-trivial contribution to the bosonic determinant. So finally, the CEP tree-level is given by:
\begin{align}\label{eq:improved_zeroOrderPot_withPhi6}
 U(m_{\Phi}, s_{\Phi}) &= -8 \kappa \left( m_{\Phi}^2 - s_{\Phi}^2 \right)   +   \left( m_{\Phi}^2 + s_{\Phi}^2 \right)
                         + \hat\lambda \left( m_{\Phi}^4 + s_{\Phi}^4 + 6 m_{\Phi}^2 s_{\Phi}^2 - 2 N_f \left(m_{\Phi}^2 + s_{\Phi}^2 \right) \right) 
         \nonumber \\
                        & + \hat\lambda_6 \left( m_{\Phi}^6 + s_{\Phi}^6 + 15 \left( m_{\Phi}^4 s_{\Phi}^2 + m_{\Phi}^2 s_{\Phi}^4 \right)   \right)
         \nonumber \\
                        & -\frac{2N_f}{V} \sum\limits_p \log
                                    \left[ \left( |\nu^+(p)| |\nu^+(\wp)|   +
                                    \hat y ^2 \left( m_{\Phi}^2 - s_{\Phi}^2 \right) |\gamma^+(p)| |\gamma^+(\wp)|\right)^2 
         \nonumber \right. \\
                        & \left. +  m_{\Phi}^2 \hat y^2 \left( |\gamma^+(p)| |\nu^+(\wp)|   -   |\nu^+(p)| |\gamma^+(\wp)| \right)^2\right]
         \nonumber \\
                        & - \frac{1}{2V}\sum\limits_{0 \neq p \neq p_s} 
          \log\left( 2 - 4 \hat\lambda N_f - 4 \kappa \sum_{\mu} \cos(p_{\mu}) + 8 \hat\lambda \left( m_{\Phi}^2 + s_{\Phi}^2  \right) 
                     +  18\, \hat\lambda_6\left( m_{\Phi}^4 + s_{\Phi}^4 + 6\, m_{\Phi}^2 s_{\Phi}^2 \right) \right)
\end{align}

The expression with the reparametrizations \eqref{eq:def_reparametrization_couplingsAndMagnetization_Nf} and \eqref{eq:lambda_6_reparametrized} for the rescaled potential
$\tilde{U}=U/N_f$ is straight foward. 
\begin{align}\label{eq:improved_zeroOrderPot_withPhi6_rescaled}
 \tilde U(\breve m_{\Phi}, \breve s_{\Phi}) &= -8 \kappa \left( \breve m_{\Phi}^2 - \breve s_{\Phi}^2 \right)   +   \left( \breve m_{\Phi}^2 + \breve s_{\Phi}^2 \right)
                         + \lambda_N \left( \breve m_{\Phi}^4 + \breve s_{\Phi}^4 + 6 \breve m_{\Phi}^2 \breve s_{\Phi}^2 
                         - 2  \left(\breve m_{\Phi}^2 + \breve s_{\Phi}^2 \right) \right) 
         \nonumber \\
                        & + \lambda_{6N} \left( \breve m_{\Phi}^6 + \breve s_{\Phi}^6 + 15 \left( \breve m_{\Phi}^4 \breve s_{\Phi}^2 
                          + \breve m_{\Phi}^2 \breve s_{\Phi}^4 \right)   \right)
         \nonumber \\
                        & -\frac{2}{V} \sum\limits_p \log
                                    \left[ \left( |\nu^+(p)| |\nu^+(\wp)|   +
                                    \hat y^2_N \left( \breve m_{\Phi}^2 - \breve s_{\Phi}^2 \right) |\gamma^+(p)| |\gamma^+(\wp)|\right)^2 
         \nonumber \right. \\
                        & \left. +  \breve m_{\Phi}^2 y^2_N \left( |\gamma^+(p)| |\nu^+(\wp)|   -   |\nu^+(p)| |\gamma^+(\wp)| \right)^2\right]
         \nonumber \\
                        & - \frac{1}{2VN_f}\sum\limits_{0 \neq p \neq p_s} 
          \log\left( 2 - 4 \lambda_N - 4 \kappa \sum_{\mu} \cos(p_{\mu}) + 8 \lambda_N \left( \breve m_{\Phi}^2 + \breve s_{\Phi}^2  \right) 
                     +  18\,\lambda_{6N} \left( \breve m_{\Phi}^4 + \breve s_{\Phi}^4 +6\, \breve  m_{\Phi}^2 \breve s_{\Phi}^2 \right) \right)
\end{align}







%
\clearpage
%
\section{\texorpdfstring{First order in $\hat\lambda$ and $\hat\lambda_6$}{First order in lambda and lambda6}}
Here I want do add the first order in $\hat\lambda$ and $\hat\lambda_6$ with the inclusion of the bosonic determinant as 
discussed in~\ref{ch:alternativeExpansion}.
In that case, the bosonic interaction part of the action is given by (collecting the necessary parts of eq.~\eqref{eq:alternative_decomposition_of_phi4_term}
and~\eqref{eq:alternative_decomposition_of_phi6_term}):
\begin{align}\label{eq:BosonicInteractionCEP_withDetAndLambda6}
 S_{B,I}  & = \frac{\hat\lambda}{V} \widetilde{\sum\limits_{p_1 \dots p_4}}\delta_{p+r,q+s} \tilde\Phi_p^{\dagger} \tilde\Phi_q \tilde\Phi_r^{\dagger} \tilde\Phi_s
     \nonumber \\
          & + \frac{\hat\lambda_6}{V} \left( 9\left( m_{\Phi}^2 + s_{\Phi}^2 \right) \right)\widetilde{\sum\limits_{p_1,\dots p4}} \delta_{p_1 + p_3, p_2 + p_4 } 
                     \tilde\Phi_{p_1}^{\dagger} \tilde\Phi_{p_2} \tilde\Phi_{p_3}^{\dagger} \tilde\Phi_{p_4}
            + \frac{\hat\lambda_6}{V^2}\widetilde{\sum\limits_{p_1,\dots p6}} \delta_{p_1 + p_3 + p_5, p_2 + p_4 + p_6}
                     \tilde\Phi_{p_1}^{\dagger} \tilde\Phi_{p_2} \tilde\Phi_{p_3}^{\dagger} \tilde\Phi_{p_4} \tilde\Phi_{p_5}^{\dagger} \tilde\Phi_{p_6},
     \nonumber \\
          & = \frac{\hat\lambda + \hat\lambda_6\left( 9\left( m_{\Phi}^2 + s_{\Phi}^2 \right) \right)}{V} \widetilde{\sum\limits_{p_1,\dots p4}} \delta_{p_1 + p_3, p_2 + p_4 } 
                     \tilde\Phi_{p_1}^{\dagger} \tilde\Phi_{p_2} \tilde\Phi_{p_3}^{\dagger} \tilde\Phi_{p_4} 
     \nonumber \\
          & + \frac{\hat\lambda_6}{V^2}\widetilde{\sum\limits_{p_1,\dots p6}} \delta_{p_1 + p_3 + p_5, p_2 + p_4 + p_6}
                     \tilde\Phi_{p_1}^{\dagger} \tilde\Phi_{p_2} \tilde\Phi_{p_3}^{\dagger} \tilde\Phi_{p_4} \tilde\Phi_{p_5}^{\dagger} \tilde\Phi_{p_6}.
\end{align}
As a reminder: The tilde above the sum indicates, that none of the momenta in the sum is the zero or staggered mode. 
The contraction of the scalar fields is given by:
\begin{equation}\label{eq:ContractionOfBosonicField_withDetAndLambda6}
 \contraction{}{\Phi}{{}^\dagger_p}{\Phi{}}  \Phi^\dagger_p \Phi_q = \frac{4 \cdot \delta_{p,q}}
                   {2 - 4 \hat\lambda N_f - 4 \kappa \sum_{\mu} \cos(p_{\mu}) 
                   + 8 \, \hat\lambda \left( m_{\Phi}^2 + s_{\Phi}^2  \right) 
                   + 18\, \hat\lambda_6\left( m_{\Phi}^4 + s_{\Phi}^4 + 6\,  m_{\Phi}^2 s_{\Phi}^2 \right) }.
\end{equation}
To clean the notation a little bit, the propagator sum will now be given by:
\begin{equation}\label{eq:bosonicPropagatorSum_CEP_withDetAndLambda6}
 \tilde P_B \equiv \frac{1}{V}\sum\limits_{0\neq p\neq p_s} \frac{1}{2 - 4 \hat\lambda N_f - 4 \kappa \sum_{\mu} \cos(p_{\mu}) 
                   + 8\,  \hat\lambda \left( m_{\Phi}^2 + s_{\Phi}^2  \right) 
                   + 18\, \hat\lambda_6\left( m_{\Phi}^4 + s_{\Phi}^4 + 6\,  m_{\Phi}^2 s_{\Phi}^2 \right)}.
\end{equation}
If one now repeats the steps leading to \eqref{eq:CEP_firstOrder}, one finds:
\begin{align}\label{eq:CEP_firstOrder_withDetAndLambda6}
 \int \left[ \prod\limits_{0 \neq p \neq p_s}  d \tilde \Phi_p \right]   
                      e^{-S_{B,I}[\Phi] +  S_{B,0}[\Phi]} 
                      = & \sqrt{\frac{V}{\det \mathcal{B}}} \cdot 
                        \left(1 - \frac{\hat\lambda + \hat\lambda_6\left( 9\left( m_{\Phi}^2 + s_{\Phi}^2 \right) \right)}{V}\left( \tilde C_{p,q} \right) 
                        - \frac{\hat\lambda_6}{V^2} \left( \tilde C_{p,q,r} \right) \right)
          \nonumber \\
                      = & \sqrt{\frac{V}{\det \mathcal{B}}} 
                        \cdot e^{\left\{ \frac{\hat\lambda + \hat\lambda_6\left( 9\left( m_{\Phi}^2 + s_{\Phi}^2 \right) \right)}{V}\left( \tilde C_{p,q} \right) 
                        - \frac{\hat\lambda_6}{V^2} \left( \tilde C_{p,q,r} \right) \right\} }.
\end{align}
The $\tilde C$ are given by:
\begin{align}\label{eq:CEP_firstOrder_twoLoop_withDetAndLambda_6}
 \tilde C_{p,q} & = \widetilde{\sum\limits_{p,q}} \left( 
                     \contraction{}{\tilde\Phi}{{}_{p}^{\dagger}}{\tilde\Phi}
                     \contraction{\tilde\Phi_{p}^{\dagger} \tilde\Phi_{p}}{\tilde\Phi}{{}_{q}^{\dagger}}{\tilde\Phi}
                     \tilde\Phi_{p}^{\dagger} \tilde\Phi_{p} \tilde\Phi_{q}^{\dagger} \tilde\Phi_{q} 
                   + \contraction{}{\tilde\Phi}{^{\dagger} \tilde\Phi_{q} \tilde\Phi_{q}^{\dagger} }{\tilde\Phi}
                     \bcontraction{\tilde\Phi_{p}^{\dagger}}{\tilde\Phi}{{}_{q}}{\tilde\Phi}
                     \tilde\Phi_{p}^{\dagger} \tilde\Phi_{q} \tilde\Phi_{q}^{\dagger} \tilde\Phi_{p}
                     \right) 
        \nonumber \\
                & = 8\, V^2\, \tilde P_B^2,
        \\\label{eq:CEP_firstOrder_threeLoop_withDetAndLambda_6}
 \tilde C_{p,q,r} & = \widetilde{\sum\limits_{p,q,r}} \left( 
                     %pp qq rr
                     \contraction{}{ \tilde\Phi}{{}_{p}^{\dagger} }{\tilde\Phi}
                     \contraction{\tilde\Phi_{p}^{\dagger} \tilde\Phi_{p}}{ \tilde\Phi}{{}_{q}^{\dagger} }{\tilde\Phi}
                     \contraction{\tilde\Phi_{p}^{\dagger} \tilde\Phi_{p}\tilde\Phi_{q}^{\dagger} \tilde\Phi_{q}}{ \tilde\Phi}{{}_{r}^{\dagger} }{\tilde\Phi}
                     \tilde\Phi_{p}^{\dagger} \tilde\Phi_{p} \tilde\Phi_{q}^{\dagger} \tilde\Phi_{q} \tilde\Phi_{r}^{\dagger} \tilde\Phi_{r} 
                     % pp qr rq
                   + \contraction{}{ \tilde\Phi}{ {}_{p}^{\dagger} }{ \tilde\Phi }
                     \bcontraction{ \tilde\Phi_{p}^{\dagger} \tilde\Phi_{p} }{ \tilde\Phi }{ {}^{\dagger} \tilde\Phi_{r} \tilde\Phi_{r}^{\dagger} }{\tilde\Phi }
                     \contraction{ \tilde\Phi_{p}^{\dagger} \tilde\Phi_{p} \tilde\Phi_{q}^{\dagger} }{ \tilde\Phi }{ {}_{r} }{ \tilde \Phi }
                     \tilde\Phi_{p}^{\dagger} \tilde\Phi_{p} \tilde\Phi_{q}^{\dagger} \tilde\Phi_{r} \tilde\Phi_{r}^{\dagger} \tilde\Phi_{q} 
%                      % pq qp rr
                   + \contraction{}{ \tilde\Phi }{ {}_{p}^{\dagger} \tilde\Phi_{q} \tilde\Phi_{q}^{\dagger} }{\tilde\Phi}
                     \bcontraction{ \tilde\Phi_{p}^{\dagger} }{ \tilde\Phi }{ {}_{q} }{\tilde \Phi}
                     \contraction{ \tilde\Phi_{p}^{\dagger} \tilde\Phi_{q} \tilde\Phi_{q}^{\dagger} \tilde\Phi_{p} }{ \tilde\Phi }{ {}_{r}^{\dagger} }{ \tilde\Phi }
                     \tilde\Phi_{p}^{\dagger} \tilde\Phi_{q} \tilde\Phi_{q}^{\dagger} \tilde\Phi_{p} \tilde\Phi_{r}^{\dagger} \tilde\Phi_{r} \right.
        \nonumber \\ 
                       % pq rp qr 
           & +\left. \contraction{}{ \tilde\Phi }{ {}_{p}^{\dagger} \tilde\Phi_{q} \tilde\Phi_{r}^{\dagger} }{ \tilde \Phi}
                     \bcontraction{ \tilde\Phi_{p}^{\dagger} }{ \tilde\Phi }{ {}_{q} \tilde\Phi_{r}^{\dagger} \tilde\Phi_{p} }{ \tilde\Phi }
                     \contraction[2ex]{ \tilde\Phi_{p}^{\dagger} \tilde\Phi_{q} }{ \tilde\Phi }{ {}_{r}^{\dagger} \tilde\Phi_{p} \tilde\Phi_{q}^{\dagger} }{ \tilde\Phi} 
                     \tilde\Phi_{p}^{\dagger} \tilde\Phi_{q} \tilde\Phi_{r}^{\dagger} \tilde\Phi_{p} \tilde\Phi_{q}^{\dagger} \tilde\Phi_{r}
                      % pq qr rp
                   + \contraction{}{ \tilde\Phi }{ {}_{p}^{\dagger} \tilde\Phi_{q} \tilde\Phi_{q}^{\dagger} \tilde\Phi_{r} \tilde\Phi_{r}^{\dagger} }{ \tilde\Phi}
                     \bcontraction{ \tilde\Phi_{p}^{\dagger} }{ \tilde\Phi }{ {}_{q} }{ \tilde\Phi }
                     \bcontraction{ \tilde\Phi_{p}^{\dagger} \tilde\Phi_{q} \tilde\Phi_{q}^{\dagger} }{ \tilde\Phi }{ {}_{r} }{ \tilde\Phi}
                     \tilde\Phi_{p}^{\dagger} \tilde\Phi_{q} \tilde\Phi_{q}^{\dagger} \tilde\Phi_{r} \tilde\Phi_{r}^{\dagger} \tilde\Phi_{p}
                      % pq rr qp
                     \contraction{}{ \tilde\Phi }{ {}_{p}^{\dagger} \tilde\Phi_{q} \tilde\Phi_{r}^{\dagger} \tilde\Phi_{r} \tilde\Phi_{q}^{\dagger} }{ \tilde \Phi }
                     \contraction[2ex]{ \tilde\Phi_{p}^{\dagger} }{ \tilde\Phi }{ {}_{q} \tilde\Phi_{r}^{\dagger} \tilde\Phi_{r} }{\tilde \Phi}
                     \bcontraction{ \tilde\Phi_{p}^{\dagger} \tilde\Phi_{q} }{ \tilde\Phi }{ {}_{r}^{\dagger} }{ \tilde\Phi }
                     \tilde\Phi_{p}^{\dagger} \tilde\Phi_{q} \tilde\Phi_{r}^{\dagger} \tilde\Phi_{r} \tilde\Phi_{q}^{\dagger} \tilde\Phi_{p}
                     \right)
        \nonumber \\
                 & = 24\, V^3\, \tilde P_B^3.
\end{align}
If one also includes the contribution coming fro the first order in $\hat\lambda$ and $\hat\lambda_6$ to the CEP \eqref{eq:improved_zeroOrderPot_withPhi6} 
or~\eqref{eq:improved_zeroOrderPot_withPhi6_rescaled}the potential
changes to:
\begin{align}\label{eq:CEP_oneLoopAddition_withDetAndLambda6}
 U(m_{\Phi}, s_{\Phi}) & \rightarrow U(m_{\Phi}, s_{\Phi}) + 8\,\left(\hat \lambda + \hat\lambda_6\left( 9\left( m_{\Phi}^2 + s_{\Phi}^2 \right) \right) \right) \tilde P_B^2
                        + 24\, \hat\lambda_6 \tilde P_B^3,
      \\ \label{eq:CEP_oneLoopAddition_withDetAndLambda6_rescaled}
 \tilde U(\breve{m}_{\Phi},\breve{s}_{\Phi}) & \rightarrow  
       \tilde U(\breve{m}_{\Phi},\breve{s}_{\Phi}) + 
                        8\,\left( \frac{\lambda_N}{N_f} + \frac{\lambda_{6N}}{N_f^2}\left( 9\left( \breve{m}_{\Phi}^2 + \breve{s}_{\Phi}^2 \right) \right) \right) \tilde P_B^2
                        + 24\, \frac{\lambda_{6N}}{N_f^2} \tilde P_B^3.
\end{align}

%
\clearpage
%
\section{The potential and its derivatives}
Here I just summerize the potential and it's derivatives for future use. Starting point is the expression \eqref{eq:improved_zeroOrderPot_withPhi6}
\begin{align}\label{eq:CEP_shortCut}
 U(m_{\Phi}, s_{\Phi}) &= U^{\text{tree}}(m_{\Phi}, s_{\Phi}) + U^{\text{ferm}}(m_{\Phi}, s_{\Phi}) + U^{\text{BosDet}}(m_{\Phi}, s_{\Phi}) 
              \\ \label{CEP_treeLevel}
 U^{\text{tree}}(m_{\Phi}, s_{\Phi}) &= -8 \kappa \left( m_{\Phi}^2 - s_{\Phi}^2 \right)   +   \left( m_{\Phi}^2 + s_{\Phi}^2 \right)
                         + \hat\lambda \left( m_{\Phi}^4 + s_{\Phi}^4 + 6 m_{\Phi}^2 s_{\Phi}^2 - 2 N_f \left(m_{\Phi}^2 + s_{\Phi}^2 \right) \right) 
              \nonumber \\
                         & + \hat\lambda_6 \left( m_{\Phi}^6 + s_{\Phi}^6 + 15 \left( m_{\Phi}^4 s_{\Phi}^2 + m_{\Phi}^2 s_{\Phi}^4 \right)   \right)
              \\ \label{eq:CEP_ferionicContribution}
 U^{\text{ferm}}(m_{\Phi}, s_{\Phi}) &= -\frac{2N_f}{V} \sum\limits_p \log
                                    \left[ \left( |\nu^+(p)| |\nu^+(\wp)|   +
                                    \hat y ^2 \left( m_{\Phi}^2 - s_{\Phi}^2 \right) g_p\right)^2 
                                              \nonumber \right. \\ 
                        & \left. +  m_{\Phi}^2 \hat y^2 \left( |\gamma^+(p)| |\nu^+(\wp)|   -   |\nu^+(p)| |\gamma^+(\wp)| \right)^2\right] 
                                              \\ \label{eq:CEP_bosonicDeterminantContribution}
 U^{\text{BosDet}}(m_{\Phi}, s_{\Phi}) &= - \frac{1}{2V}\sum\limits_{0 \neq p \neq p_s} 
                        \log\Bigl( 2 - 4 \hat\lambda N_f - 4 \kappa \sum_{\mu} \cos(p_{\mu}) + 8 \hat\lambda \left( m_{\Phi}^2 + s_{\Phi}^2  \right) 
%            \nonumber \\
                        + 18\, \hat\lambda_6\left( m_{\Phi}^4 + s_{\Phi}^4 + 6\,  m_{\Phi}^2 s_{\Phi}^2 \right)\Bigr)
\end{align}

In the follwoing the 1st and 2nd derivatives w.r.t.\ $m_{\Phi}$ and $s_{\Phi}$ are given. First the tree-level:
\begin{align}\label{CEP_derivative_dm_treeLevel}
 \frac{\partial}{\partial m_{\Phi}}  U^{\text{tree}}(m_{\Phi}, s_{\Phi})&=
                    -16 \kappa m_{\Phi} + 2 m_{\Phi} + \hat\lambda \left( 4 m_{\Phi}^3 + 12 m_{\Phi} s_{\Phi}^2 -4 N_f  m_{\Phi} \right)
                    \nonumber \\
                    & + \hat\lambda_6\left( 30 m_{\Phi}^5 + 60 m_{\Phi}^3 s_{\Phi}^2 + 30 m_{\Phi} s_{\Phi}^4 \right)
                    \\ \label{eq:CEP_derivative_ds_treeLevel}
 \frac{\partial}{\partial s_{\Phi}}  U^{\text{tree}}(m_{\Phi}, s_{\Phi})&=
                    +16 \kappa s_{\Phi} + 2 s_{\Phi} + \hat\lambda \left( 4 s_{\Phi}^3 + 12 m_{\Phi}^2 s_{\Phi} -4 N_f  s_{\Phi} \right)
                    \nonumber \\
                    & + \hat\lambda_6\left( 30 s_{\Phi}^5 + 30 m_{\Phi}^4 s_{\Phi} + 60 m_{\Phi}^2 s_{\Phi}^3 \right)
                    \\ \label{eq:CEP_derivative_dmdm_treeLevel}
 \frac{\partial^2}{\partial m_{\Phi}^2}  U^{\text{tree}}(m_{\Phi}, s_{\Phi})&=
                    -16 \kappa  + 2 + \hat\lambda \left( 12 m_{\Phi}^2 + 12 s_{\Phi}^2 - 4 N_f \right)
                    + \hat\lambda_6 \left( 30m_{\Phi}^4 + 180 m_{\Phi}^2 s_{\Phi}^2 + 30 s_{\Phi}^4 \right)
                    \\ \label{eq:CEP_derivative_dsds_treeLevel}
 \frac{\partial^2}{\partial s_{\Phi}^2}  U^{\text{tree}}(m_{\Phi}, s_{\Phi})&=
                    +16 \kappa  + 2 + \hat\lambda \left( 12 s_{\Phi}^2 + 12 m_{\Phi}^2 - 4 N_f \right)
                    + \hat\lambda_6 \left( 30m_{\Phi}^4 + 180 m_{\Phi}^2 s_{\Phi}^2 + 30 s_{\Phi}^4 \right)
 \\ \label{eq:CEP_derivative_dmds_treeLevel}
 \frac{\partial^2}{\partial m_{\Phi} \partial s_{\Phi}}  U^{\text{tree}}(m_{\Phi}, s_{\Phi})&=
                    \hat\lambda \left( 24 m_{\Phi} s_{\Phi} \right) 
                    + \hat\lambda_6 \left( 120 \left( m_{\Phi}^3 s_{\Phi} + m_{\Phi} s_{\Phi}^3 \right) \right)
\end{align}

For the derivatives of the fermionic contributions some shortcuts will be used:
\begin{align}\label{eq:CEP_ferionicContribution_shortcuts}
 U^{\text{ferm}}(m_{\Phi}, s_{\Phi}) &= -\frac{2N_f}{V} \sum\limits_p \log\left[{A_p^{m, s}}\right],
                    \\ \label{eq:CEP_ferionicContribution_shortcuts_A}
 {A_p^{m, s}} &= {a_p^{m,s}}^2 +  m_{\Phi}^2 \hat y^2 \cdot b_p^2 ,
                    \\ \label{eq:CEP_ferionicContribution_shortcuts_a}
 {a_p^{m,s}} &= |\nu^+(p)| |\nu^+(\wp)|   +   \hat y ^2 \left( m_{\Phi}^2 - s_{\Phi}^2 \right) |\gamma^+(p)| |\gamma^+(\wp)|,
                    \\ \label{eq:CEP_ferionicContribution_shortcuts_b}
 b_p                     &= |\gamma^+(p)| |\nu^+(\wp)|   -   |\nu^+(p)| |\gamma^+(\wp)|, 
                    \\ \label{eq:CEP_ferionicContribution_shortcuts_g}
 g_p                     &=  |\gamma^+(p)| |\gamma^+(\wp)|
\end{align}
so tht the they can be written in a more compact form:
\begin{align}\label{eq:CEP_derivative_dm_fermionicContribution}
 \frac{\partial }{\partial m_{\Phi}} U^{\text{ferm}}(m_{\Phi}, s_{\Phi}) &= -\frac{2N_f}{V} \sum\limits_p
                   \frac{4 m_{\Phi} \hat y^2 \cdot g_p \cdot {a_p^{m,s}}   +   2 m_{\Phi} \hat y^2 \cdot b_p^2}
                   { {A_p^{m, s}} } 
       \nonumber \\
%                    
            &= -\frac{2N_f}{V} \cdot 2 m_{\Phi} \hat y^2 \cdot \sum\limits_p 
                   \frac{2 \cdot g_p \cdot {a_p^{m,s}} + b_p^2}{ {A_p^{m, s}} }
\end{align} 
\begin{align}\label{eq:CEP_derivative_ds_fermionicContribution}
 \frac{\partial }{\partial s_{\Phi}} U^{\text{ferm}}(m_{\Phi}, s_{\Phi}) &= -\frac{2N_f}{V} \sum\limits_p
                   \frac{- 4 s_{\Phi} \hat y^2 \cdot g_p \cdot {a_p^{m,s}} }
                   { {A_p^{m, s}} } 
       \nonumber \\
%                    
            &= -\frac{2N_f}{V} \cdot \left( -4 s_{\Phi} \hat y^2\right) \cdot \sum\limits_p 
                   \frac{ g_p \cdot {a_p^{m,s}} }{ {A_p^{m, s}} }
\end{align} 
\begin{align}\label{eq:CEP_derivative_dmdm_fermionicContribution}        
 \frac{\partial^2 }{\partial m_{\Phi}^2} U^{\text{ferm}}(m_{\Phi}, s_{\Phi}) &= -\frac{2N_f}{V} \cdot 2 \hat y^2 \sum\limits_p
                   \left[ \frac{2 \cdot g_p \cdot {a_p^{m,s}} + b_p^2}{ {A_p^{m, s}} } 
                   + m_{\Phi} \cdot \frac{   4 m_{\Phi} \hat y^2 \cdot  \left(g_p\right)^2    }{{A_p^{m, s}} } \right.
       \nonumber  \\
%                    
                  &- \left. m_{\Phi} \cdot \frac{ 2 m_{\Phi} \hat y^2 \cdot \left(2 \cdot g_p \cdot {a_p^{m,s}} + b_p^2 \right)^2 }
                    { {A_p^{m, s}}^2 } \right]
       \nonumber \\
%                     
             &= -\frac{2N_f}{V} \cdot 2 \hat y^2 \sum\limits_p 
                \left[ \frac{  2 \cdot g_p \cdot {a_p^{m,s}} + b_p^2   }{ {A_p^{m, s}} } 
                  \cdot \left( 1 - \frac{2m_{\Phi}^2 \hat y ^2}{ {A_p^{m, s}} }  \right) 
                  + \frac{   4 m_{\Phi}^2 \hat y ^2 \cdot g_p^2    }{{A_p^{m, s}} }\right]
\end{align} 
\begin{align} \label{eq:CEP_derivative_dsds_fermionicContribution}
 \frac{\partial^2 }{\partial s_{\Phi}^2} U^{\text{ferm}}(m_{\Phi}, s_{\Phi}) &= -\frac{2N_f}{V} \cdot (-4 \hat y^2) \sum\limits_p
                \left[ \frac{g_p \cdot {a_p^{m,s}} }{ {A_p^{m, s}} } 
                + s_{\Phi} \frac{ -2 s_{\Phi} \hat y^2 \cdot g_p^2}{ {A_p^{m, s}} }
                - s_{\Phi} \frac{ -4 s_{\Phi} \hat y^2 \cdot \left( g_p \cdot {a_p^{m,s}} \right)^2 }{ {A_p^{m, s}}^2 } \right]
      \nonumber \\
               &= -\frac{2N_f}{V} \cdot (-4 \hat y^2) \sum\limits_p 
                  \left[ \frac{g_p}{ {A_p^{m, s}} }\cdot \left( {a_p^{m,s}} + 2 s_{\Phi}^2 \hat y ^2 g_p 
                  \cdot \left( \frac{ 2 \cdot {a_p^{m,s}}^2}{  {A_p^{m, s}} } - 1 \right) \right) \right]
\end{align}
\begin{align} \label{eq:CEP_derivative_dmds_fermionicContribution}
 \frac{\partial^2 }{\partial m_{\Phi} \partial s_{\Phi}} U^{\text{ferm}}(m_{\Phi}, s_{\Phi}) &= -\frac{2N_f}{V} \cdot (-4 s_{\Phi}\hat y^2) \sum\limits_p
                   \left[ \frac{ 2 m_{\Phi} \hat y^2 g_p^2}{ {A_p^{m, s}} } \right. 
      \nonumber \\
                 & -\left. \frac{ 2 m_{\Phi} \hat y^2 \cdot g_p \cdot {a_p^{m,s}} \cdot \left( 2 \cdot g_p\cdot {a_p^{m,s}} + b_p^2 \right)}
                     { {A_p^{m, s}}^2 } \right] 
      \nonumber \\
                &= -\frac{2N_f}{V} \cdot (-8 m_{\Phi} s_{\Phi}\hat y^4) \sum\limits_p 
                    \left[ \frac{g_p}{ {A_p^{m, s}}^2 } \cdot \left( g_p  
                    - \frac{ {a_p^{m,s}} \cdot \left( 2 \cdot g_p\cdot {a_p^{m,s}} + b_p^2 \right)}
                     { {A_p^{m, s}} }\right) \right] 
\end{align}

For the part coming from the bosonic determinant I use the shortcut:
 \begin{align}\label{eq:CEP_bosonicDeterminantContribution_shortcut}
  U^{\text{BosDet}}(m_{\Phi},s_{\Phi}) &= - \frac{1}{2V}\sum\limits_{0 \neq p \neq p_s} \log \left[ D_p^{m,s} \right]
       \nonumber \\
   D_p^{m,s}                                 &=  2 - 4 \hat\lambda N_f - 4 \kappa \sum_{\mu} \cos(p_{\mu}) + 8\, \hat \lambda \left(m_{\Phi}^2 + s_{\Phi}^2 \right) 
                                        + 18\, \hat\lambda_6\left( m_{\Phi}^4 + s_{\Phi}^4 + 6\,  m_{\Phi}^2 s_{\Phi}^2 \right).
 \end{align}
Since $U^{\text{BosDet}}$ is completely symmetric in $m_{\Phi}$ and $s_{\Phi}$, so I'll only write the ones w.r.t.\ $m_{\Phi}$:
\begin{align}\label{eq:CEP_derivative_dm_bosonicContribution}
 \frac{\partial }{\partial m_{\Phi}} U^{\text{BosDet}}(m_{\Phi}, s_{\Phi}) &= - \frac{1}{2V}\sum\limits_{0 \neq p \neq p_s} 
                   \left[  \frac{16\, \hat\lambda\,  m_{\Phi}  + 72\,\hat\lambda_6 \left( m_{\Phi}^3 + 3\, m_{\Phi} s_{\Phi}^2 \right) }
                   { D_p^{m,s} }  \right]
       \\ \label{eq:CEP_derivative_dmdm_bosonicContribution}
%        
 \frac{\partial^2 }{\partial m_{\Phi}^2} U^{\text{BosDet}}(m_{\Phi}, s_{\Phi}) &= - \frac{1}{2V}\sum\limits_{0 \neq p \neq p_s} 
                   \left[  \frac{16\, \hat\lambda + 216\,\hat\lambda_6\left( m_{\Phi}^2 + s_{\Phi}^2 \right) }{ D_p^{m,s} }
                   - \left( \frac{16\, \hat\lambda\, m_{\Phi}  + 72\,\hat\lambda_6 \left( m_{\Phi}^3 + 3\, m_{\Phi} s_{\Phi}^2 \right)}{D_p^{m,s} } \right)^2    \right]
              \\ \label{eq:CEP_derivative_dmds_bosonicContribution}
%               
 \frac{\partial^2 }{\partial m_{\Phi}\partial s_{\Phi}} U^{\text{BosDet}}(m_{\Phi}, s_{\Phi}) &= - \frac{1}{2V}\sum\limits_{0 \neq p \neq p_s} 
                   \left[ \frac{432\, \hat\lambda_6\, m_{\Phi} s_{\Phi} }{ D_p^{m,s} } \right.
        \nonumber \\
                   & - \left.\frac{\left(16\, \hat\lambda\,  m_{\Phi}  + 72\,\hat\lambda_6 \left( m_{\Phi}^3 + 3\, m_{\Phi} s_{\Phi}^2 \right) \right)
                    \cdot \left(16\, \hat\lambda\,  s_{\Phi}  + 72\,\hat\lambda_6 \left( s_{\Phi}^3 + 3\, m_{\Phi}^2 s_{\Phi} \right) \right) }
                    { {D_p^{m,s}}^2 }  \right]
\end{align}

The corresponding expression for the rescaled potential are easily obtained by noting that:
\begin{equation}\label{eq:derivativesOfRescaledPotential}
 \frac{\partial \tilde U}{\partial \breve m_{\Phi}} = \frac{1}{\sqrt{N_f}}\frac{\partial U}{\partial m_{\Phi}},\qquad
 \frac{\partial^2 \tilde U}{\partial \breve m_{\Phi}^2} = \frac{\partial^2 U}{\partial m_{\Phi}^2}.
\end{equation}
Explicitly, for the three parts, to get the expressions, just exchange the couplings by their rescaled counterparts and
\begin{itemize}
 \item For $U^{\text{tree}}$, remove the explicit appearence of $N_f$,
 \item for $U^{\text{ferm}}$, remove the $N_f$ in the numerator n front of the sum,
 \item for $U^{\text{BosDet}}$, add a factor of $N_f^{-1}$ in front of the sum.
\end{itemize}



%
\clearpage
%
\section{The potential and its derivatives in the rescaled formulation}
To get rid of some confusion, I will here write the corresponding equations to those from \ref{ch:potAndDerivatives} with the 
reparametrizations~\eqref{eq:def_reparametrization_couplingsAndMagnetization_Nf} and~\eqref{eq:lambda_6_reparametrized} included. The potential itself is then given by:
\begin{align}\label{eq:CEP_shortCut_rescaled}
 \tilde U(\breve m_{\Phi}, \breve s_{\Phi}) &= 
                            \tilde U^{\text{tree}}(\breve m_{\Phi}, \breve s_{\Phi}) + \tilde U^{\text{ferm}}(\breve m_{\Phi}, \breve s_{\Phi})
                          + \tilde U^{\text{BosDet}}(\breve m_{\Phi}, \breve s_{\Phi}) + \tilde U^{\text{1st}}(\breve m_{\Phi}, \breve s_{\Phi})
%                           
              \\ \label{CEP_treeLevel_rescaled}
 \tilde U^{\text{tree}}(\breve m_{\Phi}, \breve s_{\Phi}) & = 
                          - 8\, \kappa \left( \breve m_{\Phi}^2 - \breve s_{\Phi}^2 \right)   +   \left( \breve m_{\Phi}^2 + \breve s_{\Phi}^2 \right)
                          + \lambda_N \left( \breve m_{\Phi}^4 + \breve s_{\Phi}^4 + 6 \breve m_{\Phi}^2 \breve s_{\Phi}^2 
                          - 2 \left(\breve m_{\Phi}^2 + \breve s_{\Phi}^2 \right) \right) 
              \nonumber \\
                        & + \lambda_{6N} \left( \breve m_{\Phi}^6 + \breve s_{\Phi}^6 + 15 \left( \breve m_{\Phi}^4 \breve s_{\Phi}^2 
                          + \breve m_{\Phi}^2 \breve s_{\Phi}^4 \right)   \right)
%                           
              \\ \label{eq:CEP_ferionicContribution_rescaled}
 \tilde U^{\text{ferm}}(\breve m_{\Phi}, \breve s_{\Phi}) & = 
                          - \frac{2}{V} \sum\limits_p \log \left[ \left( |\nu^+(p)| |\nu^+(\wp)|   +
                           y_N^2 \left( \breve m_{\Phi}^2 - \breve s_{\Phi}^2 \right) g_p\right)^2  \right.
               \nonumber \\ 
                        & + \left. \breve m_{\Phi}^2 y_N^2 \left( |\gamma^+(p)| |\nu^+(\wp)|   -   |\nu^+(p)| |\gamma^+(\wp)| \right)^2\right] 
%                         
               \\ \label{eq:CEP_bosonicDeterminantContribution_rescaled}
 \tilde U^{\text{BosDet}}(\breve m_{\Phi}, \breve s_{\Phi}) & = 
                          - \frac{1}{2\,N_f\,V}\sum\limits_{0 \neq p \neq p_s} 
                            \log\Bigl[ 2 - 4 \lambda_N - 4 \kappa \sum_{\mu} \cos(p_{\mu})
               \nonumber \\
                        & + 8 \lambda_N \left( \breve m_{\Phi}^2 + \breve s_{\Phi}^2  \right) 
                          + 18\, \lambda_{6N}\left( \breve m_{\Phi}^4 + \breve s_{\Phi}^4 + 6\,  \breve m_{\Phi}^2 \breve s_{\Phi}^2 \right)\Bigr]
%                           
               \\ \label{eq:CEP_bosonicFirstOrderContribution_rescaled}
 \tilde U^{\text{1st}}(\breve m_{\Phi}, \breve s_{\Phi}) & = 
                            \frac{8}{N_f^2} \left( \lambda_N + \lambda_{6N} \left( 
                            9 \left( \breve m_{\Phi}^2 + \breve s_{\Phi}^2 \right) \right) \right) \tilde P_B^2 + 24\,\frac{\lambda_{6N}}{N_f^3} \tilde P_B^3,
\end{align}


\subsection*{Tree-level}
\begin{align}\label{CEP_derivative_dm_treeLevel_rescaled}
 \partial_{\breve m_{\Phi}} \tilde U^{\text{tree}}(\breve m_{\Phi}, \breve s_{\Phi})& =
                           - 16 \,\kappa\, \breve m_{\Phi} + 2\, \breve m_{\Phi} 
                           + \lambda_N \left( 4\, \breve m_{\Phi}^3 + 12 \,\breve m_{\Phi} \, \breve s_{\Phi}^2 - 4\, \breve m_{\Phi} \right)
         \nonumber \\
                         & + \lambda_{N6}\left( 30\, \breve m_{\Phi}^5 + 60\, \breve m_{\Phi}^3\, \breve s_{\Phi}^2 + 30\, \breve m_{\Phi}\, \breve s_{\Phi}^4 \right)
%                     
         \\ \label{eq:CEP_derivative_ds_treeLevel_rescaled}
 \partial_{\breve s_{\Phi}}   \tilde U^{\text{tree}}(\breve m_{\Phi}, \breve s_{\Phi}) & =
                    +16 \,\kappa\, \breve s_{\Phi} + 2\, \breve s_{\Phi} 
                    + \lambda_N \left( 4\, \breve s_{\Phi}^3 + 12\, \breve m_{\Phi}^2\, \breve s_{\Phi} -4\, \breve s_{\Phi} \right)
         \nonumber \\
                  & + \lambda_{N6}\left( 30\, \breve s_{\Phi}^5 + 30\, \breve m_{\Phi}^4\, \breve s_{\Phi} + 60\, \breve m_{\Phi}^2\, \breve s_{\Phi}^3 \right)
%          
         \\ \label{eq:CEP_derivative_dmdm_treeLevel_rescaled}
 \left(\partial_{\breve m_{\Phi}}\right)^2   \tilde U^{\text{tree}}(\breve m_{\Phi}, \breve s_{\Phi}) & =
                    - 16\, \kappa  + 2 + \lambda_N \left( 12\, \breve m_{\Phi}^2 + 12\, \breve s_{\Phi}^2 - 4 \right)
                    + \lambda_{N6} \left( 30\, \breve m_{\Phi}^4 + 180\, \breve m_{\Phi}^2\, \breve s_{\Phi}^2 + 30\, \breve s_{\Phi}^4 \right)
% 
        \\ \label{eq:CEP_derivative_dsds_treeLevel_rescaled}
 \left(\partial_{\breve s_{\Phi}} \right)^2  \tilde U^{\text{tree}}(\breve m_{\Phi}, \breve s_{\Phi}) & =
                    +16\, \kappa  + 2 + \lambda_N \left( 12\, \breve s_{\Phi}^2 + 12\, \breve m_{\Phi}^2 - 4 \right)
                    + \lambda_{N6} \left( 30\, \breve m_{\Phi}^4 + 180\, \breve m_{\Phi}^2\, \breve s_{\Phi}^2 + 30\, \breve s_{\Phi}^4 \right)
%                     
        \\ \label{eq:CEP_derivative_dmds_treeLevel_rescaled}
 \partial_{\breve m_{\Phi}} \partial_{\breve s_{\Phi}}   \tilde U^{\text{tree}}(\breve m_{\Phi}, \breve s_{\Phi}) & =
                    \lambda_N \left( 24\, \breve m_{\Phi}\, \breve s_{\Phi} \right) 
                    + \lambda_{N6} \left( 120 \left( \breve m_{\Phi}^3\, \breve s_{\Phi} + \breve m_{\Phi}\, \breve s_{\Phi}^3 \right) \right)
\end{align}

\subsection*{Fermionic contribution}
For the derivatives of the fermionic contributions the shortcuts change slightly and $N_f$ does not appear in the overall numerator anymore. Further, the number 
for $A_p^{m,s}$ is th same, since the rescaling of $\hat y$ and $m_{\Phi}/s_{\Phi}$ cancel:
\begin{align}\label{eq:CEP_ferionicContribution_shortcuts_rescaled}
 \tilde U^{\text{ferm}}(\breve m_{\Phi}, \breve s_{\Phi}) & = - \frac{2}{V} \sum\limits_p \log\left[{A_p^{m, s}}\right],
%  
          \\ \label{eq:CEP_ferionicContribution_shortcuts_A_rescaled}
 {A_p^{m, s}} &= {a_p^{m,s}}^2 +  \breve m_{\Phi}^2 \hat y^2 \cdot b_p^2 ,
%  
          \\ \label{eq:CEP_ferionicContribution_shortcuts_a_rescaled}
 {a_p^{m,s}} &= |\nu^+(p)| |\nu^+(\wp)|   +   \hat y ^2 \left( \breve m_{\Phi}^2 - \breve s_{\Phi}^2 \right) |\gamma^+(p)| |\gamma^+(\wp)|,
\end{align}
For $b_p$ and $g_p$ nothing changes at all.
% 
\begin{align}\label{eq:CEP_derivative_dm_fermionicContribution_rescaled}
 \partial_{\breve m_{\Phi}}  \tilde U^{\text{ferm}}(\breve m_{\Phi}, \breve s_{\Phi}) & =
                                -\frac{2}{V} \sum\limits_p \frac{4 \breve m_{\Phi} \hat y^2 \cdot g_p \cdot {a_p^{m,s}}  
                                +  2 \breve m_{\Phi} \hat y^2 \cdot b_p^2} { {A_p^{m, s}} } 
       \nonumber \\
                              & = - \frac{2}{V} \cdot 2 \breve m_{\Phi} \hat y^2 \cdot \sum\limits_p 
                                  \frac{2 \cdot g_p \cdot {a_p^{m,s}} + b_p^2}{ {A_p^{m, s}} }
\end{align} 
% 
\begin{align}\label{eq:CEP_derivative_ds_fermionicContribution_rescaled}
 \partial_{\breve s_{\Phi}}  \tilde U^{\text{ferm}}(\breve m_{\Phi}, \breve s_{\Phi}) &= 
                              - \frac{2}{V} \sum\limits_p \frac{- 4 \breve s_{\Phi} \hat y^2 \cdot g_p \cdot {a_p^{m,s}} }{ {A_p^{m, s}} } 
       \nonumber \\
                            & = - \frac{2}{V} \cdot \left( -4 \breve s_{\Phi} \hat y^2\right) \cdot \sum\limits_p 
                                \frac{ g_p \cdot {a_p^{m,s}} }{ {A_p^{m, s}} }
\end{align} 
% 
\begin{align}\label{eq:CEP_derivative_dmdm_fermionicContribution_rescaled}        
 \left(\partial_{\breve m_{\Phi}} \right)^2 \tilde U^{\text{ferm}}(\breve m_{\Phi}, \breve s_{\Phi}) & = 
                                  - \frac{2}{V} \cdot 2 \hat y^2 \sum\limits_p \left[ \frac{2 \cdot g_p \cdot {a_p^{m,s}} + b_p^2}{ {A_p^{m, s}} } 
                                  + \breve m_{\Phi} \cdot \frac{   4 \breve m_{\Phi} \hat y^2 \cdot  \left(g_p\right)^2    }{{A_p^{m, s}} } \right.
       \nonumber \\
                                & - \left. \breve m_{\Phi} \cdot \frac{ 2 \breve m_{\Phi} \hat y^2 \cdot \left(2 \cdot g_p \cdot {a_p^{m,s}} + b_p^2 \right)^2 }
                                    { {A_p^{m, s}}^2 } \right]
       \nonumber \\
                                & = - \frac{2}{V} \cdot 2 \hat y^2 \sum\limits_p \left[ \frac{  2 \cdot g_p \cdot {a_p^{m,s}} + b_p^2   }{ {A_p^{m, s}} } 
                                    \cdot \left( 1 - \frac{2\breve m_{\Phi}^2 \hat y ^2}{ {A_p^{m, s}} }  \right) 
                                  + \frac{   4 \breve m_{\Phi}^2 \hat y ^2 \cdot g_p^2    }{{A_p^{m, s}} }\right]
\end{align} 
% 
\begin{align} \label{eq:CEP_derivative_dsds_fermionicContribution_rescaled}
 \left(\partial_{\breve s_{\Phi}} \right)^2 \tilde U^{\text{ferm}}(\breve m_{\Phi}, \breve s_{\Phi}) & = 
                                   - \frac{2}{V} \cdot (-4 \hat y^2) \sum\limits_p \left[ \frac{g_p \cdot {a_p^{m,s}} }{ {A_p^{m, s}} } 
                                  + \breve s_{\Phi} \frac{ -2 \breve s_{\Phi} \hat y^2 \cdot g_p^2}{ {A_p^{m, s}} }
                                  - \breve s_{\Phi} \frac{ -4 \breve s_{\Phi} \hat y^2 \cdot \left( g_p \cdot {a_p^{m,s}} \right)^2 }{ {A_p^{m, s}}^2 } \right]
      \nonumber \\
                                & = -\frac{2}{V} \cdot (-4 \hat y^2) \sum\limits_p \left[ \frac{g_p}{ {A_p^{m, s}} }\cdot \left( {a_p^{m,s}} 
                                    + 2 \breve s_{\Phi}^2 \hat y ^2 g_p  \cdot \left( \frac{ 2 \cdot {a_p^{m,s}}^2}{  {A_p^{m, s}} } - 1 \right) \right) \right]
\end{align}
% 
\begin{align} \label{eq:CEP_derivative_dmds_fermionicContribution_rescaled}
 \partial_{\breve m_{\Phi}} \partial_{\breve s_{\Phi}}  \tilde U^{\text{ferm}}(\breve m_{\Phi}, \breve s_{\Phi}) & = 
                                    - \frac{2}{V} \cdot (-4 \breve s_{\Phi}\hat y^2) \sum\limits_p \left[ \frac{ 2 \breve m_{\Phi} \hat y^2 g_p^2}{ {A_p^{m, s}} } \right. 
      \nonumber \\
                                  & - \left. \frac{ 2 \breve m_{\Phi} \hat y^2 \cdot g_p \cdot {a_p^{m,s}} \cdot \left( 2 \cdot g_p\cdot {a_p^{m,s}} + b_p^2 \right)}
                                      { {A_p^{m, s}}^2 } \right] 
      \nonumber \\
                                  & = - \frac{2}{V} \cdot (-8 \breve m_{\Phi} \breve s_{\Phi}\hat y^4) \sum\limits_p 
                                      \left[ \frac{g_p}{ {A_p^{m, s}}^2 } \cdot \left( g_p  - \frac{ {a_p^{m,s}} \cdot \left( 2 \cdot g_p\cdot {a_p^{m,s}} + b_p^2 \right)}
                                      { {A_p^{m, s}} }\right) \right] 
\end{align}
% 
% 
\subsection*{Bosonic determinant}
Here, for the shortcut $D_p^{m,s}$ nothing changes, but the replacement, but its derivatives change, so let's redefine it:
 \begin{align}\label{eq:CEP_bosonicDeterminantContribution_shortcut_rescaled}
  \tilde U^{\text{BosDet}}(\breve m_{\Phi},\breve s_{\Phi}) &= - \frac{1}{2\,N_f\,V}\sum\limits_{0 \neq p \neq p_s} \log \left[ \tilde D_p^{m,s} \right],
       \nonumber \\
   \tilde D_p^{m,s}  &=  2 - 4 \lambda_N - 4 \kappa \sum_{\mu} \cos(p_{\mu}) + 8\, \lambda_N \left(\breve m_{\Phi}^2 + \breve s_{\Phi}^2 \right) 
                                        + 18\, \lambda_{6N}\left( \breve m_{\Phi}^4 + \breve s_{\Phi}^4 + 6\,  \breve m_{\Phi}^2 \breve s_{\Phi}^2 \right).
%                                         
       \\ \label{eq:CEP_derivativeShortcutD_dm_rescaled}
   \partial_{\breve m_{\Phi}}  \tilde D_p^{m,s} & = 
                                          16\, \lambda_N\,  \breve m_{\Phi}  + 72\,\lambda_{6N} \left( \breve m_{\Phi}^3 + 3\, \breve m_{\Phi} \breve s_{\Phi}^2 \right),
%                                           
         \\ \label{eq:CEP_derivativeShortcutD_dmdm_rescaled}
 \left(\partial_{\breve m_{\Phi}}\right)^2  \tilde D_p^{m,s} & = 
                                          16\, \lambda_N + 216\,\lambda_{6N} \left( \breve m_{\Phi}^2 + \breve s_{\Phi}^2 \right),
%                                           
         \\ \label{eq:CEP_derivativeShortcutD_dmds_rescaled}
 \partial_{\breve m_{\Phi}} \partial_{\breve s_{\Phi}} \tilde D_p^{m,s} & = 432\,\lambda_{6N} \, \breve m_{\Phi}\, \breve s_{\Phi}.
 \end{align}
%  
For the generalized Propagatorsums nothing special happens. Formally, it looks completely equal:
\begin{align}\label{eq:def_powersOfPropSumWithDerivative_rescaled}
 \tilde P_{B^n} & \equiv \frac{1}{V}\sum\limits_{0\neq p\neq p_s} \left(\frac{1}{ \tilde D_p^{m,s}}\right)^n
           \\ \label{eq:derivative_Of_P_Bn_rescaled}
 \partial_{\breve m_{\Phi}/\breve s_{\Phi}} \tilde P_{B^n} & = n \left( \partial_{\breve m_{\Phi}/\breve s_{\Phi}} \tilde D_p^{m,s} \right) \tilde P_{B^{n+1}},
\end{align}
% 
% 
\begin{align}\label{eq:CEP_derivative_dm_bosonicContribution_rescaled}
 \partial_{\breve m_{\Phi}}  \tilde U^{\text{BosDet}}(\breve m_{\Phi}, \breve s_{\Phi}) & = 
                                  - \frac{1}{2\,N_f} \left( \partial_{\breve m_{\Phi}} \tilde D_p^{m,s} \right) \tilde P_B 
%         
        \\ \label{eq:CEP_derivative_dmdm_bosonicContribution_rescaled}
 \left(\partial_{\breve m_{\Phi}} \right)^2 \tilde U^{\text{BosDet}}(\breve m_{\Phi}, \breve s_{\Phi}) & = 
                                  - \frac{1}{2\,N_f}\left[ \left( \left(\partial_{\breve m_{\Phi}}\right)^2 \tilde D_p^{m,s} \right) \, \tilde P_B
                                  - \left( \partial_{\breve m_{\Phi}} \tilde D_p^{m,s} \right)^2 \, \tilde P_{B^2}    \right]
%               
         \\ \label{eq:CEP_derivative_dmds_bosonicContribution_rescaled}
\partial_{\breve m_{\Phi}} \partial_{\breve s_{\Phi}}  \tilde U^{\text{BosDet}}(\breve m_{\Phi}, \breve s_{\Phi}) & = 
                                  - \frac{1}{2\,N_f} \left[ \left( \partial_{\breve m_{\Phi}} \partial_{\breve s_{\Phi}} \tilde D_p^{m,s} \right) \tilde P_B 
                                  - \left( \partial_{\breve m_{\Phi}}  \tilde D_p^{m,s} \right)\left( \partial_{\breve s_{\Phi}}  \tilde D_p^{m,s} \right)\tilde P_{B^2} \right]
\end{align}
% 
\subsection*{First order Contribution}
Here some factors of $N_f$ appear:
\begin{align}\label{eq:CEP_firstWithDet_shortcuts_rescaled}
 \tilde U^{\text{1st}}(\breve m_{\Phi}, \breve s_{\Phi}) & \equiv \tilde \alpha^{m, s}\, \tilde P_B^2 + \tilde \beta\, \tilde P_B^3,
%  
        \\ \label{CE:CEP_firstWithDet_shrotcut_alpha_rescaled}
 \tilde \alpha^{m, s}  & = \frac{8}{N_f^2} \left(\lambda_N + \lambda_{6N}\left( 9\left( \breve m_{\Phi}^2 + \breve s_{\Phi}^2 \right) \right) \right)
%  
        \\ \label{CE:CEP_firstWithDet_shrotcut_dm_alpha_rescaled}
 \partial_{\breve m_{\Phi}} \tilde \alpha^{m, s}  & = 144\, \frac{\lambda_{6N}}{N_f^2}\, \breve m_{\Phi}
%  
        \\ \label{CE:CEP_firstWithDet_shrotcut_dmdm_alpha_rescaled}
 \left(\partial_{\breve m_{\Phi}}\right)^2 \tilde \alpha^{m, s}  & = 144\, \frac{\lambda_{6N}}{N_f^2}
%  
        \\ \label{CE:CEP_firstWithDet_shrotcut_beta_rescaled}
 \tilde \beta  & = 24\, \frac{\lambda_{6N}}{N_f^3}
\end{align}

Since the first order contribution is symmetric in $\breve m_{\Phi}$ and $\breve s_{\Phi}$, only the derivatives w.r.t.\ $\breve m_{\Phi}$are given. 
\begin{align}\label{eq:CEP_derivative_dm_firstOrder_rescaled}
 \partial_{\breve m_{\Phi}} \tilde U^{\text{1st}}(\breve m_{\Phi}, \breve s_{\Phi}) & = 
                           144\,\frac{\lambda_{6N}}{N_f^2}\, \breve m_{\Phi}\, \tilde P_B^2 
                         - 2\,\tilde \alpha^{m,s}\left( \partial_{\breve m_{\Phi}} \tilde D_p^{m,s} \right) \tilde P_B\, \tilde P_{B^2}
                         - 3\, \tilde \beta \left( \partial_{\breve m_{\Phi}} \tilde D_p^{m,s} \right) \tilde P_B^2\,  \tilde P_{B^2}
%                          
       \\ \label{eq:CEP_derivative_dmdm_firstOrder_rescaled}
 \left(\partial_{\breve m_{\Phi}}\right)^2 \tilde U^{\text{1st}}(\breve m_{\Phi}, \breve s_{\Phi}) & = 
                           144\,\frac{\lambda_{6N}}{N_f^2}\, \tilde P_B^2 -
                           \left[ 288\,\frac{\lambda_{6N}}{N_f^2}\,\breve m_{\Phi}\left( \partial_{\breve m_{\Phi}} \tilde D_p^{m,s} \right) 
                         + 2\,\tilde \alpha^{m,s}\left( \left(\partial_{\breve m_{\Phi}}\right)^2 \tilde D_p^{m,s} \right) \right] \tilde P_B\, \tilde P_{B^2}
       \nonumber \\
                       & + 2\,\tilde \alpha^{m,s}\left( \partial_{\breve m_{\Phi}} \tilde D_p^{m,s} \right)^2 \left[ \tilde P_{B^2}^2 + 2\, \tilde P_B\, \tilde P_{B^3} \right]
       \nonumber \\
                       & - 3\, \tilde \beta \left( \left(\partial_{\breve m_{\Phi}}\right)^2 \tilde D_p^{m,s} \right) \tilde P_B^2\,  \tilde P_{B^2}
                         + 6\, \tilde \beta \left( \partial_{\breve m_{\Phi}} \tilde D_p^{m,s} \right)^2 
                         \left[ \tilde P_B\,  \tilde P_{B^2}^2 - \tilde P_B^2\,  \tilde P_{B^3} \right]
%                          
      \\ \label{eq:CEP_derivative_dmds_firstOrder_rescaled}
 \partial_{\breve m_{\Phi}}\partial_{\breve s_{\Phi}}\tilde U^{\text{1st}}(\breve m_{\Phi}, \breve s_{\Phi}) & =
                         \left[- 288\,\frac{\lambda_{6N}}{N_f^2}\left( \breve m_{\Phi} \left( \partial_{\breve s_{\Phi}} \tilde D_p^{m,s} \right) + \breve s_{\Phi} \left( \partial_{\breve m_{\Phi}} \tilde D_p^{m,s} \right) \right)
                         - 2\,\tilde \alpha^{m,s} \left( \partial_{\breve m_{\Phi}} \partial_{\breve s_{\Phi}} \tilde D_p^{m,s} \right) \right] \tilde P_B\, \tilde P_{B^2}
      \nonumber \\
                       & + 2\,\tilde \alpha^{m,s} \left( \partial_{\breve m_{\Phi}} \tilde D_p^{m,s} \right)\left( \partial_{\breve s_{\Phi}} \tilde D_p^{m,s} \right)
                         \left[ \tilde P_{B^2}^2 + 2\, \tilde P_B\, \tilde P_{B^3} \right]
      \nonumber \\
                       & - 3\, \tilde \beta \left( \partial_{\breve m_{\Phi}}\partial_{\breve s_{\Phi}} \tilde D_p^{m,s} \right) \tilde P_B^2\,  \tilde P_{B^2}
                         + 6\, \tilde \beta \left( \partial_{\breve m_{\Phi}} \tilde D_p^{m,s} \right)\left( \partial_{\breve s_{\Phi}} \tilde D_p^{m,s} \right)
                         \left[ \tilde P_B\,  \tilde P_{B^2}^2 - \tilde P_B^2\,  \tilde P_{B^3} \right]
\end{align}


%
\clearpage
%
\section{Lower mass bound from the CEP}
To determin the lower Higgs boson mass in the CEP one cann assume to be in the broken phase wie non-zero
magnetization and a staggered mode being zero. Further one can decompose the scalar into a Higgs mode ($h$) and three 
Goldstone modes ($g^{\alpha}, \alpha=1,2,3$). Further we are interested in the effect, the addition of a $\lambda_6 \cdot (\phi^{\dagger}\phi)^3)$ term
might have on this bound.

In this chaper we use the continuum parameters ($m_0^2, \lambda, \lambda_6$), however, they are still considered as dimensionless.

In this approach, the cutoff is an input and $m_0^2$ is tuned to obtain a minimum at the desired value of the field. The potential is given by:
\begin{multline}\label{eq:CEP_lowerBound_with_phi_6}
 U(\hat \phi) = U_f(\hat \phi) + \frac{m_0^2}{2} {\hat \phi}^2 +\lambda {\hat \phi}^4 + \lambda_6 {\hat \phi}^6 \\
                         + \lambda \cdot {\hat \phi}^2 \cdot 6(P_H+P_G)
                         + \lambda_6 \cdot \left( {\hat \phi}^2 \cdot ( 45 P_H^2 + 54 P_G P_H + 45 P_G^2)
                         + {\hat \phi}^4 \cdot ( 15 P_H + 9 P_G ) \right ),
\end{multline}
with:
\begin{align}
 \label{eq:fermionic_contribution_CEP_massbound}
 U_f(\hat \phi) =& -\frac{2\, N_f}{V} \left[
                 \sum\limits_p \log\left| \nu(p) + y_t \cdot \hat \phi \cdot \left( 1-\frac{1}{2 \rho} \right) \nu(p) \right|^2 +
                 \sum\limits_p \log\left| \nu(p) + y_b \cdot \hat \phi \cdot \left( 1-\frac{1}{2 \rho} \right) \nu(p) \right|^2 \right] \\
 \label{eq:def_propagator_sums}
 P_{H/G} =& \frac{1}{V} \sum\limits_{p \neq 0} \frac{1}{{\hat p}^2 + m_{H/G}^2}
\end{align}

With the cutoff $\Lambda$ being fixed, the location of the minimum is fixed and $m_0^2$ has to be coosen according to:
\begin{align}
 \left. \frac{\text{d} U}{\text{d}\hat \phi} \right|_{\hat \phi = v} \stackrel{!}{=}& 0, \\
 U'(\hat \phi)=&  U_f'(\hat \phi) + m_0^2 {\hat \phi} + 4 \lambda {\hat \phi}^3 + 6 \lambda_6 {\hat \phi}^5 + \nonumber \\
            & + \lambda \cdot {\hat \phi} \cdot 12 (P_H + P_G) 
              + \lambda_6 \cdot \left( 2 {\hat \phi} \cdot ( 45 P_H^2 + 54 P_G P_H + 45 P_G^2)
                         + 4{\hat \phi}^3 \cdot ( 15 P_H + 9 P_G ) \right ) \\
%  
 \Rightarrow m_0^2 =& -\frac{U_f'(v)}{v} - 4 \lambda {v}^2 - 6 \lambda_6 {v}^4 + \nonumber \\
                    & \lambda \cdot 12 (P_H + P_G)
                       -\lambda_6 \cdot \left( 2 \cdot ( 45 P_H^2 + 54 P_G P_H + 45 P_G^2)
                         + 4 v^2 \cdot ( 15 P_H + 9 P_G ) \right ) \\
%                          
 U(\hat \phi) =& U_f(\hat \phi) - 2\frac{U_f'(v)}{v} {\hat \phi}^2
                 + \lambda \left( {\hat \phi}^4 - 2 {\hat \phi}^2( v^2)  \right) \nonumber \\
               & + \lambda_6 \left( {\hat \phi}^6 + {\hat \phi}^4 ( 15 P_H + 9 P_G ) 
                 - 2 v^2 {\hat \phi}^2 ( 15 P_H + 9 P_G ) - 3 v^4 {\hat \phi}^2 \right)                 
\end{align}

The Higgs boson mass $m_H$ is then obtained by the curvature of the potential in its minimum:
\begin{align}
 U''({\hat\phi}) &= U_f''({\hat\phi}) + m_0^2 + 12 \lambda {\hat\phi}^2 + 30 \lambda_6 {\hat\phi}^4 + \nonumber \\
                 & + \lambda \cdot  \left( 12 (P_H + P_G) \right)
                   + \lambda_6 \cdot \left( 2 (45 P_H^2 + 54 P_H P_G + 45 P_G^2) + 12 {\hat\phi}^2 (15 P_H + 9 P_G) \right)
\end{align}




%
\clearpage
%
\section{Phase structure and Higgs mass in the broken phase}
Since it turned out, that the approach where one assumes to be in te broken phase with the potential given in~\eqref{eq:CEP_lowerBound_with_phi_6} works pretty well in 
determining the phase structure in presence of small but negative  $\lambda$ and small positive $\lambda_6$ we continue from this approach. Since there are still 
deviation between simulation data and prediction from the CEP, David suggested, to use a similar approach as in chapter~\ref{ch:alternativeExpansion}, where the gaussian 
contributions from the interaction part are actually considered for the gaussian part with the drawback of having a vev-dependent bosonic determinant and more complicated 
expressions for the propagators. However, in this ansatz no iterative scheme is needed, since the Higgs mass should only be taken from the curvature of the potential in its
minimum. Still, the minimization has to be performed, which will be slightly more complicated.

Starting point here is the bosonic action, where the scalar field was already decomposed into Higgs and Goldstone modes. The only difference in treating those two 
kinds of fields is in the assumtion of the zero mode. For the Higgs field it is proportional to the vev, while it is zero for the Goldstones:
\begin{equation}\label{eq:def_zeromodes_inBrokenPhase}
 \tilde h_0 = \sqrt{V} \hat v,\quad\quad \tilde g_0^{\alpha}=0.
\end{equation}
The action is:
\begin{align}\label{eq:bosonic_action_in_broken_phase_momentum}
 S & = \frac{1}{2}\sum\limits_{p} \left \{ \tilde h_{-p} \left( \hat p^2 + m_0^2 \right) \tilde h_{p} + 
                \sum\limits_{\alpha} \tilde g^{\alpha}_{-p} \left( \hat p^2 + m_0^2 \right) \tilde g^{\alpha}_{p} \right\}
        \nonumber \\
    & + \frac{\lambda}{V} \sum \limits_{p_1 \dots p_4} \delta_{p_1+ \dots +p_4,0} 
                \left( \tilde h_{p_1} \tilde h_{p_2} + \sum\limits_{\alpha} \tilde g^{\alpha}_{p_1} \tilde g^{\alpha}_{p_2} \right)
                \left( \tilde h_{p_3} \tilde h_{p_4} + \sum\limits_{\alpha} \tilde g^{\alpha}_{p_3} \tilde g^{\alpha}_{p_4} \right)
             \nonumber \\
    & + \frac{\lambda_6}{V^2} \sum \limits_{p_1 \dots p_6} \delta_{p_1+ \dots +p_6,0} 
                \left( \tilde h_{p_1} \tilde h_{p_2} + \sum\limits_{\alpha} \tilde g^{\alpha}_{p_1} \tilde g^{\alpha}_{p_2} \right)
                \left( \tilde h_{p_3} \tilde h_{p_4} + \sum\limits_{\alpha} \tilde g^{\alpha}_{p_3} \tilde g^{\alpha}_{p_4} \right)
                \left( \tilde h_{p_5} \tilde h_{p_6} + \sum\limits_{\alpha} \tilde g^{\alpha}_{p_5} \tilde g^{\alpha}_{p_6} \right).
\end{align}

With this, a decomposition of this action can be done by decomposing it into a tree-level, a gaussian and an interaction part. Tree-level is easy, simply collect
all terms that only contain zero modes. 
\begin{equation}\label{eq:tree_level_action_inBrokenPhase}
 S_B^{\text{tree}} = V \left( \frac{m_0^2}{2} \hat v^2 + \lambda \hat v^4 + \lambda_6 \hat v ^6 \right).
\end{equation}
For th gaussian term, we collect all terms, where all but two fields are set to their zero modes. This should lead to:
\begin{equation}\label{eq:improvedGaussian_action_inBrokenPhase}
 S_B^{\text{gauss}} = \frac{1}{2}\sum\limits_{p \neq 0} \left\{
                 \tilde h_{-p} \left ( \hat p^2 + m_0^2 + 12 \hat v^2 \lambda + 30 \hat v^4 \lambda_6 \right) \tilde h_p
                 + \sum\limits_{\alpha} 
                 \tilde g^{\alpha}_{-p} \left ( \hat p^2 + m_0^2 + 4 \hat v^2 \lambda + 6 \hat v^4 \lambda_6 \right) \tilde g^{\alpha}_p
                 \right\}
\end{equation}
The interaction part is simply given by the rest:
\begin{align}\label{eq:improvedInteraction_action_inBrokenPhase}
 S_B^{\text{int}} & = \frac{\lambda}{V}\widetilde{\sum\limits_{p_1\dots p_4}} \delta_{p_1+ \dots +p_4,0} 
                      \left( \tilde h_{p_1} \tilde h_{p_2} + \sum\limits_{\alpha} \tilde g^{\alpha}_{p_1} \tilde g^{\alpha}_{p_2} \right)
                      \left( \tilde h_{p_3} \tilde h_{p_4} + \sum\limits_{\alpha} \tilde g^{\alpha}_{p_3} \tilde g^{\alpha}_{p_4} \right)
                      \nonumber \\
                  & + \frac{\lambda_6}{V}\hat v^2 \widetilde{\sum\limits_{p_1\dots p_4}} \delta_{p_1+ \dots +p_4,0} \left\{
                    + 30 \, \tilde h_{p_1} \tilde h_{p_2} \tilde h_{p_3} \tilde h_{p_4} 
                    + 18 \, \tilde h_{p_1} \tilde h_{p_2} \sum\limits_{\alpha} \tilde g^{\alpha}_{p_3} \tilde g^{\alpha}_{p_4}
                    + 3  \, \sum\limits_{\alpha,\beta} \tilde g^{\alpha}_{p_1} \tilde g^{\alpha}_{p_2}\tilde g^{\beta}_{p_3} \tilde g^{\beta}_{p_4}
                 \right\}
                      \nonumber \\
                  & + \frac{\lambda_6}{V^2} \widetilde {\sum \limits_{p_1 \dots p_6}} \delta_{p_1+ \dots +p_6,0} 
                \left( \tilde h_{p_1} \tilde h_{p_2} + \sum\limits_{\alpha} \tilde g^{\alpha}_{p_1} \tilde g^{\alpha}_{p_2} \right)
                \left( \tilde h_{p_3} \tilde h_{p_4} + \sum\limits_{\alpha} \tilde g^{\alpha}_{p_3} \tilde g^{\alpha}_{p_4} \right)
                \left( \tilde h_{p_5} \tilde h_{p_6} + \sum\limits_{\alpha} \tilde g^{\alpha}_{p_5} \tilde g^{\alpha}_{p_6} \right).
\end{align}
From the gaussian part~\eqref{eq:improvedGaussian_action_inBrokenPhase}, we get the propagators:
\begin{equation}
    \label{eq:prop_HiggsAndGoldstone_improved_inBrokenPhase}
 \contraction{}{\tilde h}{{}_p}{\tilde h{}}  \tilde h_p \tilde h_q = 
                              \frac{ \delta_{p+q,0}}{\hat p^2 + m_0^2 + 12 \hat v^2 \lambda + 30 \hat v^4 \lambda_6 } \quad \text{and} \quad
 \contraction{}{\tilde g}{{}^{\alpha}_p}{\tilde g{}}  \tilde g^{\alpha}_p \tilde g^{\beta}_q = 
                              \frac{ \delta_{p+q,0} \delta_{\alpha,\beta}}{\hat p^2 + m_0^2 + 4 \hat v^2 \lambda + 6 \hat v^4 \lambda_6 }.
\end{equation}
As before, for convenience let's define the propagator sums for later use:
\begin{equation}\label{eq:propagatorSum_improved_inBrokenPhase}
\tilde P_H = \frac{1}{V} \sum\limits_{p\neq 0} \frac{1}{ \hat p^2 + m_0^2 + 12 \hat v^2 \lambda + 30 \hat v^4 \lambda_6 } \quad\text{and}\quad
\tilde P_H = \frac{1}{V} \sum\limits_{p\neq 0} \frac{1}{ \hat p^2 + m_0^2 + 4 \hat v^2 \lambda + 6 \hat v^4 \lambda_6 }.
\end{equation}

In this approach, the bosonic determinant has to be considered for the CEP as well as the first order contributions in $\lambda$ and $\lambda_6$
\begin{align}
      \label{eq:CEP_improved_inBrokenPhase_all}
 U(\hat v) & = U^F(\hat v) + U^T(\hat v) + U^D(\hat v) + U^{(1)}(\hat v) 
            \\
      \label{eq:CEP_improved_inBrokenPhase_Fermion}
 U^F(\hat v) & = -\frac{2\, N_f}{V} \sum\limits_p \left\{
                  \log\left| \nu(p) + y_t \cdot \hat v \cdot \left( 1-\frac{1}{2 \rho} \right) \nu(p) \right|^2 +
                  \log\left| \nu(p) + y_b \cdot \hat v \cdot \left( 1-\frac{1}{2 \rho} \right) \nu(p) \right|^2 \right\}
            \\
      \label{eq:CEP_improved_inBrokenPhase_Tree}
 U^T(\hat v) & = \frac{m_0^2}{2} \hat v^2 + \lambda \hat v^4 + \lambda_6 \hat v^6
             \\
      \label{eq:CEP_improved_inBrokenPhase_BosDet}
 U^D(\hat v) & = \sum\limits_{p \neq 0} \left\{
                  \log \left( \hat p^2 + m_0^2 + 12 \hat v^2 \lambda + 30 \hat v^4 \lambda_6 \right) + 
                  3\, \log \left( \hat p^2 + m_0^2 + 4 \hat v^2 \lambda + 6 \hat v^4 \lambda_6 \right) \right\}
\end{align}

% \bibliographystyle{unsrt}
% \addcontentsline{toc}{Chapter}{Bibliography}
% \clearpage
% \bibliography{references_Higgs_Yukawa.bib}
%
\end{document}