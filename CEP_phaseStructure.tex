To determine the phase structure of the model, i.e.\ determine the phase the system is in depending on the choice of parameters, 
one has to find out, what the minimum of the CEP is w.r.t.\ $m_{\Phi}$ and $s_{\Phi}$. The simplest ansatz, would be the tree-level (again with fermions):
\begin{align}\label{eq:CEP_treelevel}
 U(m_{\Phi}, s_{\Phi}) =& -8 \kappa \left( m_{\Phi}^2 - s_{\Phi}^2 \right)   +   \left( m_{\Phi}^2 + s_{\Phi}^2 \right)
                         + \hat\lambda \left( m_{\Phi}^4 + s_{\Phi}^4 + 6 m_{\Phi}^2 s_{\Phi}^2 - 2 N_f \left(m_{\Phi}^2 + s_{\Phi}^2 \right) \right) 
                           \nonumber \\
                        & -\frac{2N_f}{V} \sum\limits_p \log
                                    \left[ \left( |\nu^+(p)| |\nu^+(\wp)|   +
                                    \hat y ^2 \left( m_{\Phi}^2 - s_{\Phi}^2 \right) |\gamma^+(p)| |\gamma^+(\wp)|\right)^2 
                                    \nonumber \right. \\ 
                        & \left. +  m_{\Phi}^2 \hat y^2 \left( |\gamma^+(p)| |\nu^+(\wp)|   -   |\nu^+(p)| |\gamma^+(\wp)| \right)^2\right]
\end{align}
Here $\nu^+(p)$ labels the eigenvalue of the overlap operator\footnote{In the definition of this eigenvalue and how it is implemeted in the 
code, there is a difference in the prefactor of $r$, the Wilson parameter. I think, its origin may lie in an ambiguity of
the definition of the Wilson parameter. What is written hiere, is how it is in the code and how I implemented it}. 
It expression and the other abreviations used are ($a=1$):
\begin{equation}
 \label{eq:eigenvalue_overlapOperator_and_gamma}
 \nu^{\pm}(p) = \rho \left( 1 + 
                      \frac{ \pm i \sqrt{{\tilde p} ^2} + r  {\hat p}^2 - \rho}
                           {\sqrt{ {\tilde p} ^2 + \left( r  {\hat p}^2 - \rho\right)^2}}\right),\qquad
 \gamma^{\pm}(p) = 1- \frac{1}{2 \rho}\nu^{\pm}(p),
 \end{equation}
 \begin{equation}
 \label{eq:def_momentumAbreviations}
 {\hat p}^2 = 4 \sum\limits_{\mu} \sin^2\left(\frac{p_{\mu}}{2}\right),\qquad
 {\tilde p}^2 = \sum\limits_{\mu} \sin^2\left(p_{\mu}\right),\qquad
 \wp = p + p_s.
\end{equation}

This is however just a very crude approximation. In Philipp's thesis, he draw a common factor of $N_f$ in front of the action and redefined the couplings,
magnetizations and the field according to:
\begin{equation}\label{eq:def_reparametrization_couplingsAndMagnetization_Nf}
 \hat y = \frac{y_N}{\sqrt{N_f}},\qquad
 \hat \lambda = \frac{\lambda_N}{N_f},\qquad
 \breve{m}_{\Phi} = \sqrt{N_f} m_{\Phi},\qquad
 \breve{s}_{\Phi} = \sqrt{N_f} s_{\Phi},\qquad 
 \breve{\Phi}=\frac{\Phi.}{\sqrt{N_f}}.
\end{equation}
The reason for that is, that the action becomes independet of $N_f$. 
When keeping now $\lambda_N$ and $y_N$ constant when performing the limit $N_f \rightarrow \infty$, everything but the treelevel get's suppressed by 
powers of $\frac{1}{N_f}$.
To reproduce the expression used by Philipp, also draw a factor in front of $U$:
\begin{align}\label{eq:CEP_treelevel_reparametrized}
 \tilde U(\breve{m}_{\Phi},\breve{s}_{\Phi}) =& \frac{U}{N_f} \nonumber \\
            =& -8 \kappa \left( \breve{m}_{\Phi}^2 - \breve{s}_{\Phi}^2 \right)   +   \left( \breve{m}_{\Phi}^2 + \breve{s}_{\Phi}^2 \right)
                         + \lambda_N \left( \breve{m}_{\Phi}^4 + s_{\Phi}^4 + 6 \breve{m}_{\Phi}^2 \breve{s}_{\Phi}^2
                         - 2 \left(\breve{m}_{\Phi}^2 + s_{\Phi}^2 \right) \right) 
                           \nonumber \\
                        & -\frac{2}{V} \sum\limits_p \log
                                    \left[ \left( |\nu^+(p)| |\nu^+(\wp)|   +
                                    \hat y ^2 \left( \breve{m}_{\Phi}^2 - \breve{s}_{\Phi}^2 \right) |\gamma^+(p)| |\gamma^+(\wp)|\right)^2 
                                    \nonumber \right. \\ 
                        & \left. +  \breve{m}_{\Phi}^2 y_N^2 \left( |\gamma^+(p)| |\nu^+(\wp)|   -   |\nu^+(p)| |\gamma^+(\wp)| \right)^2\right].
\end{align}

The addition of the first order term in $\hat \lambda$ is straight foward:
\begin{equation}\label{eq:CEP_oneLoopAddition}
 U(m_{\Phi}, s_{\Phi}) \rightarrow U(m_{\Phi}, s_{\Phi}) + 16\, \hat \lambda\, \left(m_{\Phi}^2+s_{\Phi}^2\right) P_B, \qquad 
 \tilde U(\breve{m}_{\Phi},\breve{s}_{\Phi}) \rightarrow  
       \tilde U(\breve{m}_{\Phi},\breve{s}_{\Phi}) + 16 \,\frac{\lambda_N}{N_f}\,\left( \breve{m}_{\Phi}^2 + \breve{s}_{\Phi}^2 \right) P_B.
\end{equation}





\subsection*{Limits of the CEP and problems with first order in $\lambda$}
The whole approach of integrating out the Gaussian contribution is only valid, if the bosonic determinant~\eqref{eq:def_BosonicDeterminant}
is positive definit. This limits the range of values for $\hat\lambda$ and $\kappa$:
\begin{equation}\label{eq:limits_for_CEP_from_matrix}
 0 \stackrel{!}{<} 2 - 4 N_f - 4 \kappa \sum\limits_{\mu} \cos(p_{\mu}) \quad \Rightarrow \quad
 \hat\lambda < \frac{1}{2 N_f} ,\qquad
 |\kappa| < \frac{1 - 2 N_f \hat \lambda}{8}.
\end{equation}
While this might only be a somehow estitical problem for the tree-level, it gets serious when including the the first order in 
$\lambda$ (\eqref{eq:CEP_firstOrderLambda_zeroModeContribution} and \eqref{eq:CEP_firstOrderLambda_staggeredModeContribution}), since the denominator
becomes negative or (close to) zero, spoiling everything.
This is extremely unpleasant, since the first order in $\lambda$ has a huge effect. To see this, remove the fermions and assume, you are in a phase, where 
there should be either the symmetric or the ferromagnetic phase, i.e.\ $s_{\Phi}=0$. Then the CEP reduces to:
\begin{equation}\label{eq:simple_CEP_reduction}
 U(m_{\Phi}) = \Bigl( \underbrace{1 - 8\kappa - 2 N_f \hat \lambda}_{\mathcal{O}(0\dots 2)} 
             + \underbrace{ 16\, \hat\lambda\, P_B}_{\mathcal{O}(10\hat\lambda)} \Bigr)\cdot m_{\Phi}^2 + \hat \lambda m_{\Phi}^4.
\end{equation}
For this potential, to have a minimum at $m_{\Phi}\neq 0$, the coefficient of the quadratic term must be negative. Without the first order term, this happens 
when $\kappa$ violates the condition~\eqref{eq:limits_for_CEP_from_matrix}. With fermions turned on, they help, that a non-zero minimum 
is found without violating this condition. However, the effect of the fermions is only small. If one now compares the order of magnitude, the first order term in 
$\hat\lambda$ give a very strong contribution which cannot be compensated by the fermions anymore (with resonable couplings at least).


