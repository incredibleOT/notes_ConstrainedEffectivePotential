To compute the CEP perturbatively, one decomposes the action into a part, that only depends on the assumed ground state ($S_{B,\Phi^g}$), 
a gaussian part ($S_{B,0}$) where the fields only appear quadratically and an interaction part ($S_I$) that consiste of the rest. The treelevel part 
part can be computed directly \footnote{The computation of the fermionic contribution is somehow complicated, but in principle 
free of any perturbative arguments \textbf{check that}}.  The gaussian part can be integrated out. It gives a determinant and determines the propagators 
of the bosonic field. The determinant will be ignored in the following, since for what we need it does not matter, since it is indeppendent of $m$ and $s$.
The decomposition of the bosonic part is:
\begin{align}
 S_{B,\Phi^g} &= V \left( -8 \kappa \left( m_{\Phi}^2 - s_{\Phi}^2 \right)  +  m_{\Phi}^2 + s_{\Phi}^2 
             + \hat\lambda \left( m_{\Phi}^4 + s_{\Phi}^4 + 6 m_{\Phi}^2s_{\Phi}^2 - 2 N_f \left(m_{\Phi}^2 + s_{\Phi}^2\right) \right) \right) ,
             \label{eq:BosonicTreelevelCEP} \\
 S_{B,0}      &= \frac{1}{2}  \sum\limits_{0\neq p \neq p_s}  \tilde\Phi_p^{\dagger}  
              \left[ 2 - 4 \hat\lambda N_f - 4 \kappa \sum_{\mu} \cos(p_{\mu}) \right]  \tilde\Phi_p ,
              \label{eq:BosonicGaussianCEP} \\
 S_{B,I}        &= \frac{\hat\lambda}{V} \widehat{\sum\limits_{p,q,r,s}}\delta_{p+r,q+s} \tilde\Phi_p^{\dagger} \tilde\Phi_q \tilde\Phi_r^{\dagger} \tilde\Phi_s,
              \label{eq:BosonicInteractionCEP}
\end{align}
with $\hat\sum$ meaning, that at leas one of the summed momenta is not in $\{0,p_s\}$. For the prefactor ($6 m_{\Phi}^2s_{\Phi}^2$) in the treelevel 
expression one has to take into account, that the $\delta_{p,q}$ has to respect the periodicity, meaning that $p$ and $q$ are equal up to a 
difference of multiples of $2\pi$.

From $S_0$ on gets the contraction of the scala field:
\begin{equation}\label{eq:ContractionOfBosonicField}
 \contraction{}{\Phi}{{}^\dagger_p}{\Phi{}}  \Phi^\dagger_p \Phi_q = \frac{4 \cdot \delta_{p,q}}{2 - 4 \hat\lambda N_f - 4 \kappa \sum_{\mu} \cos(p_{\mu}) }.
\end{equation}
The factor 4 in the numerator comes from the fact, that this contraction is performed on every component of the scalar field. \textbf{Since this is a crucial statement, find someone to check this!}

For the exponential of the CEP we can write now (including the fermionic Part, that will contribute):

\begin{equation} \label{eq:expandedCEP}
 e^{-V\cdot U(m,s)} = e^{-S_{B,\Phi^g}[\Phi^g]} e^{-N_f \log \det \mathcal{M}[\Phi^g]} 
                      \int \left. \left[\prod\limits_{0 \neq p \neq p_s}  d \tilde \Phi_p \right]   
                      e^{-S_{B,I}[\Phi] +  S_{B,0}[\Phi]}
                       \right|_{ \begin{array}{l} \scriptscriptstyle \tilde \Phi_0=\sqrt{V} m_{\Phi} \\ 
                                     \scriptscriptstyle \tilde \Phi_{p_s}=\sqrt{V}s_{\Phi} \end{array}},
\end{equation}
If one expands the exponential from $S_{B,I}$ to first order, the integral in \eqref{eq:expandedCEP} we end up with:
\begin{align}\label{eq:CEP_firstOrder}
 \int \left[ \prod\limits_{0 \neq p \neq p_s}  d \tilde \Phi_p \right]   
                      e^{-S_{B,I}[\Phi] +  S_{B,0}[\Phi]} =& \int \left[ \prod\limits_{0 \neq p \neq p_s}  d \tilde \Phi_p \right] 
                      \left( 1 - \frac{\hat\lambda}{V} \widehat{\sum\limits_{p,q,r,s}}\delta_{p+r,q+s} 
                      \tilde\Phi_p^{\dagger} \tilde\Phi_q \tilde\Phi_r^{\dagger} \tilde\Phi_s \right)e^{-S_B,0} \nonumber \\
                      =& \sqrt{\frac{V}{\det \mathcal{B}}} \cdot \left(1 - \frac{\hat\lambda}{V} \left( C_{p,q} + C_{m,p} + C_{s,p} \right) \right) \nonumber \\
                      =& \sqrt{\frac{V}{\det \mathcal{B}}} \cdot e^{-\frac{\hat\lambda}{V} \left( C_{p,q} + C_{m,p} + C_{s,p} \right)}
\end{align}
In the above equations, $\det\mathcal{B}$ is the bosonic determinant. The matrix $\mathcal{B}$ is given by:
\begin{equation}\label{eq:def_BosonicDeterminant}
\mathcal{B}^{i,j}(p,q)=\delta_{i,j}\delta_{p,q} \left( 2 - 4 \hat\lambda N_f - 4 \kappa \sum\limits_{\mu} \cos(p_{\mu}) \right)
\end{equation}
\textbf{Check, if determinant occurs in the correct power!}
The indices $i$ and $j$ label the components of the scalar field. It is independet of $m_{\Phi}/s_{\Phi}$ and will, 
when the log of the r.h.s. of \eqref{eq:CEP_firstOrder} is taken, only contributa a constant and can be neglected.
The $C$ come from the different possibilities to contract the four fields. 
The first one, $C_{p,q}$ comes from a full contraction:
\begin{equation}\label{eq:CEP_firstOrderLambda_VacuumBubble}
 \contraction{}{\tilde\Phi}{{}_p^{\dagger}}{\tilde\Phi} \tilde\Phi_p^{\dagger} \tilde\Phi_q
  \contraction{}{\tilde\Phi}{{}_r^{\dagger}}{\tilde\Phi} \tilde\Phi_r^{\dagger} \tilde\Phi_s = 
    \frac{ \delta_{p,q}}{2 - 4 \hat\lambda N_f - 4 \kappa \sum_{\mu} \cos(p_{\mu}) } \cdot 
    \frac{ \delta_{r,s}}{2 - 4 \hat\lambda N_f - 4 \kappa \sum_{\mu} \cos(r_{\mu}) },
\end{equation}
and gives some kind of vacuum bubble which is independent of zero or staggered mode 
and will therefore be neglected here.
The other two contributions in the exponantial origin from those terms in the sum, where some (but not all) occuring momenta are either $0$ or $p_s$:
\begin{align}\label{eq:CEP_firstOrderLambda_zeroModeContribution}
 C_{p,m}=& \widehat{\sum\limits_{p,q}} \left( 
             \contraction{}{\Phi}{{}_p^{\dagger}}{\Phi}                              \Phi_p^{\dagger} \Phi_q \Phi_0^{\dagger} \Phi_0
           + \contraction{}{\Phi}{{}_p^{\dagger} \Phi_0 \Phi_0^{\dagger}}{\Phi}      \Phi_p^{\dagger} \Phi_0 \Phi_0^{\dagger} \Phi_q
           + \contraction{\Phi_0^{\dagger} }{\Phi}{{}_p}{\Phi}                       \Phi_0^{\dagger} \Phi_p \Phi_q^{\dagger} \Phi_0
           + \contraction{\Phi_0^{\dagger} \Phi_0 }{\Phi}{{}_p^{\dagger}}{\Phi}      \Phi_0^{\dagger} \Phi_0 \Phi_p^{\dagger} \Phi_q
            \right) \nonumber \\
        =& 16 \cdot V \cdot m_{\Phi}^2 \sum\limits_{0\neq p \neq p_s} \frac{1}{2 - 4 \hat\lambda N_f - 4 \kappa \sum_{\mu} \cos(p_{\mu})} \\
  \label{eq:CEP_firstOrderLambda_staggeredModeContribution}
 C_{p,s}=& 16 \cdot V \cdot s_{\Phi}^2 \sum\limits_{0\neq p \neq p_s} \frac{1}{2 - 4 \hat\lambda N_f - 4 \kappa \sum_{\mu} \cos(p_{\mu})}.
\end{align}
For later use, the propagator sum will be defined:
\begin{equation}\label{eq:bosonicPropagatorSum_CEP}
 P_B \equiv \frac{1}{V}\sum\limits_{0\neq p\neq p_s} \frac{1}{2 - 4 \hat\lambda N_f - 4 \kappa \sum_{\mu} \cos(p_{\mu})}.
\end{equation}


