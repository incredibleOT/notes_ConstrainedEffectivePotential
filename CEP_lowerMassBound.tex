To determin the lower Higgs boson mass in the CEP one cann assume to be in the broken phase wie non-zero
magnetization and a staggered mode being zero. Further one can decompose the scalar into a Higgs mode ($h$) and three 
Goldstone modes ($g^{\alpha}, \alpha=1,2,3$). Further we are interested in the effect, the addition of a $\lambda_6 \cdot (\phi^{\dagger}\phi)^3)$ term
might have on this bound.

In this chaper we use the continuum parameters ($m_0^2, \lambda, \lambda_6$), however, they are still considered as dimensionless.

In this approach, the cutoff is an input and $m_0^2$ is tuned to obtain a minimum at the desired value of the field. The potential is given by:
\begin{multline}\label{eq:CEP_lowerBound_with_phi_6}
 U(\hat \phi) = U_f(\hat \phi) + \frac{m_0^2}{2} {\hat \phi}^2 +\lambda {\hat \phi}^4 + \lambda_6 {\hat \phi}^6 \\
                         + \lambda \cdot {\hat \phi}^2 \cdot 6(P_H+P_G)
                         + \lambda_6 \cdot \left( {\hat \phi}^2 \cdot ( 45 P_H^2 + 54 P_G P_H + 45 P_G^2)
                         + {\hat \phi}^4 \cdot ( 15 P_H + 9 P_G ) \right ),
\end{multline}
with:
\begin{align}
 \label{eq:fermionic_contribution_CEP_massbound}
 U_f(\hat \phi) =& -\frac{2\, N_f}{V} \left[
                 \sum\limits_p \log\left| \nu(p) + y_t \cdot \hat \phi \cdot \left( 1-\frac{1}{2 \rho} \right) \nu(p) \right|^2 +
                 \sum\limits_p \log\left| \nu(p) + y_b \cdot \hat \phi \cdot \left( 1-\frac{1}{2 \rho} \right) \nu(p) \right|^2 \right] \\
 \label{eq:def_propagator_sums}
 P_{H/G} =& \frac{1}{V} \sum\limits_{p \neq 0} \frac{1}{{\hat p}^2 + m_{H/G}^2}
\end{align}

With the cutoff $\Lambda$ being fixed, the location of the minimum is fixed and $m_0^2$ has to be coosen according to:
\begin{align}
 \left. \frac{\text{d} U}{\text{d}\hat \phi} \right|_{\hat \phi = v} \stackrel{!}{=}& 0, \\
 U'(\hat \phi)=&  U_f'(\hat \phi) + m_0^2 {\hat \phi} + 4 \lambda {\hat \phi}^3 + 6 \lambda_6 {\hat \phi}^5 + \nonumber \\
            & + \lambda \cdot {\hat \phi} \cdot 12 (P_H + P_G) 
              + \lambda_6 \cdot \left( 2 {\hat \phi} \cdot ( 45 P_H^2 + 54 P_G P_H + 45 P_G^2)
                         + 4{\hat \phi}^3 \cdot ( 15 P_H + 9 P_G ) \right ) \\
%  
 \Rightarrow m_0^2 =& -\frac{U_f'(v)}{v} - 4 \lambda {v}^2 - 6 \lambda_6 {v}^4 + \nonumber \\
                    & \lambda \cdot 12 (P_H + P_G)
                       -\lambda_6 \cdot \left( 2 \cdot ( 45 P_H^2 + 54 P_G P_H + 45 P_G^2)
                         + 4 v^2 \cdot ( 15 P_H + 9 P_G ) \right ) \\
%                          
 U(\hat \phi) =& U_f(\hat \phi) - 2\frac{U_f'(v)}{v} {\hat \phi}^2
                 + \lambda \left( {\hat \phi}^4 - 2 {\hat \phi}^2( v^2)  \right) \nonumber \\
               & + \lambda_6 \left( {\hat \phi}^6 + {\hat \phi}^4 ( 15 P_H + 9 P_G ) 
                 - 2 v^2 {\hat \phi}^2 ( 15 P_H + 9 P_G ) - 3 v^4 {\hat \phi}^2 \right)                 
\end{align}

The Higgs boson mass $m_H$ is then obtained by the curvature of the potential in its minimum:
\begin{align}
 U''({\hat\phi}) &= U_f''({\hat\phi}) + m_0^2 + 12 \lambda {\hat\phi}^2 + 30 \lambda_6 {\hat\phi}^4 + \nonumber \\
                 & + \lambda \cdot  \left( 12 (P_H + P_G) \right)
                   + \lambda_6 \cdot \left( 2 (45 P_H^2 + 54 P_H P_G + 45 P_G^2) + 12 {\hat\phi}^2 (15 P_H + 9 P_G) \right)
\end{align}



